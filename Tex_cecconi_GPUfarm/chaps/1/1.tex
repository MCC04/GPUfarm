
\chapter{Introduction} % (fold)

	\label{chap:intro}
	%\setcounter{intro}{1}
	This is the first section.
	\pagenumbering{arabic}
	\section{Goals}
The main goal of this thesis is to study GPU's behavior when used for different purposes with respect to the common ones.
In particular, we wanted to use a GPU to perform a code that comes closer to a \textit{\textbf{stream parallel pattern}}.
Then we observed ongoings, in terms of  \textit{completion time} and \textit{speed up }.
We now see in detail the concepts we've just introduced.

	\subsection{GPU Architecture and Data Parallel}
	\textbf{Graphics Processing Unit} (\textit{GPU}) is a coprocessor, generally known as a highly parallel multiprocessor optimized for visual computing.
	Compared with multicore CPUs, manycore GPUs have a different architectural design point, one focused on executing many parallel threads efficiently on many cores.
	This is achieved using simpler  cores  and  optimizing  for  data parallel  behavior  among  groups  of  threads, so more  of  the  per-chip  transistor  budget  is  devoted to computation \cite{pattersonhennessy}.
	
	In most of situations, indeed visual processing can be associated to a data parallel pattern.
	In general, we can roughly think to an image as a given and known amount of data upon which we want to do some computations. Obviously, once the proper granularity of the problem has been chosen, this work should be done for each 
	portion of the image.
	Given that a GPU has to process huge amount of data, we wish to have a lot of threads (lot of cores consequently) doing "the same things" on all data portions. 
	
	\subsection{Other Applications}
	However in recent years we're moving to \textbf{General-purpose computing on graphics processing units} (\textit{GPGPUs}).
	In other words, lately GPUs have been used for other applications than graphics processing,  
	
	\subsection{GP-GPUs and Stream Parallel}
	
	\section{Expectations}
	
	\section{Results}
	
	\section{Tools}
	
	
% chapter intro (end)