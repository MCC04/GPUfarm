
\chapter{Introduction}
\pagenumbering{arabic}
\label{chap:intro}
In a scenario were image processing needed to get more and more sophisticated, we saw \textit{graphic processor} follow the change getting increasingly powerful, not only in computation speed but also in flexibility.

The new elasticity given to \textbf{GPU}s made possible to exploit their benefits for a wide range of non-image-processing problems. This is the beginning of \textbf{GPGPU}s.

Despite this, enthusiasm has been slowed down when scientific community had to deal with problems that seemed to be unsuitable for GPGPUs.But when the going gets tough, the tough get going and several studies and researches showed some good results and possibilities.

This gives oxygen to keep trying mapping to GPU apparently unsuitable problems and that's the core of this work too.

\pagenumbering{arabic}
\section{Goals}
	The main goal of this thesis is to study GPU's behavior when used for different purposes with respect to the common ones.
	In particular, we wanted to use a GPU to perform a code that comes closer to a \textit{\textbf{stream parallel pattern}}.	
	Then we observed ongoings, in terms of  \textit{completion time} and \textit{speed up }.
	We now see in detail the concepts we've just introduced.

\subsection{GPU Architecture and Data Parallel}
	\textbf{GPU} (\textbf{\textit{Graphics Processing Unit}}) is a co-processor, generally known as a highly parallel multiprocessor optimized for visual computing.
	Compared with multicore CPUs, manycore GPUs have different architectural design points, one focused on executing many parallel threads efficiently on many cores.
	This is achieved using simpler cores and  optimizing for data parallel behavior among  groups of threads, so more of the per-chip  transistor budget is devoted to computation \cite{pattersonhennessy}.
	
	In most of situations, visual processing can be associated to a \textbf{\textit{data parallel pattern}}.
	In general, we can roughly think to an image as a given and known amount of \textit{independent} data upon which we want to do some computations. In most of cases, once the proper granularity of the problem has been chosen, this work should be done for each portion of the image.
	Considering the above scenario and given that generally a GPU should have to process huge amount of data, we wish to have a lot of threads (lot of cores consequently) doing "the same things" on all data portions.
	
	And that's why GPUs performs their best on data parallel problems. 

\subsection{Other Applications}
\label{subs:otherApps}
	However in recent years we're moving to \textbf{GPGPUs} (\textbf{\textit{General-purpose computing on graphics processing units}}).
	In other words, lately GPUs have been used for other applications than graphics processing.
	
	
	One of the first attempts of non-graphical computations on a GPU was a matrix-matrix multiply. In 2001, low-end graphics cards had no floating-point support; floating-point color buffers arrived in 2003.
	For the scientific community, the addition of floating point, meant no more problems on fixed-point arithmetic overflow. 
	
	Other computational advances were possible thanks to programmable shaders, that broke the rigidity of the fixed graphics pipeline (for example LU factorization with partial pivoting on a GPU was one of the first common kernels, that ran faster than an optimized CPU implementation).
	
	The introduction of \textbf{NVIDIA}’s \textbf{CUDA} (\textbf{\textit{Compute Unified Device Architecture}}) in 2007, ushered a new era of improved performance for many applications as programming GPUs became simpler: terms such as texels, fragments, and pixels were superseded with \textit{threads}, \textit{vector processing}, \textit{data caches} and \textit{shared memory} \cite{fromCUtoOCL}. 
	
	In our work we took advantage of CUDA features  \footnote{We'll show some further informations about CUDA in \hyperref[sect:tools]{Section 1.4}.}. \\
	
	One thing we should point out from GPGPUs birth: initially scientific applications on GPGPUs started from matrix (or vector) computations, that mainly could be referred to as \textbf{\textit{data parallel problems}}.
	But over time scientific community felt the need to cover other applications, that not necessarily fit data parallel model.
	
	In particular some of latest researches are moving towards \textbf{\textit{Task parallel}} applications (sometimes also known as \textit{Irregular-Workloads parallel patterns}).\\
	
	An example of non-data parallel problem is the \textit{backtracking paradigm}.
	It's oftentimes at the core of compute-and-memory-intensive problems and we can find its application in: constraint satisfaction in AI, maximal clique enumeration in graph mining, k-d tree traversal for ray tracing in graphics.
	
	Some computational motifs perform effectively on a GPU, while the effectiveness of others is still an open issue. 		
	In several studies it was highlighted that memory-bound algorithms on the GPU perform at the same level or worse than the corresponding CPU implementation. 
	
	In particular a task-parallel system should:
	\begin{itemize}
		\item Handle divergent workflows;
		\item Handle irregular parallelism;
		\item Respect dependencies between tasks;
		\item Load balance all of this.
	\end{itemize}
	
	Those requirements can lead to inefficient use of the GPU memory hierarchy and SIMD-optimized GPU multi-processors.
	
	However, there have been backtracking-based or other task-parallel algorithms successfully mapped onto the GPU: the most visible example is in \textit{ray tracing} rendering technique; another is \textit{H.264 Intra Prediction} video compression encoding; \textit{Reyes Rendering}; Deferred Lighting.
		
	But, in general, we	cannot expect an order of magnitude increase in performance. Rather, a more realistic goal is to perform at one-two times the CPU performance, which opens up the possibility of building future non-data-parallel algorithms on heterogeneous hardware (such as CPU-GPU clusters) and performing workload-based optimizations	\cite{backtrack}.
 

\subsection{GP-GPUs and Stream Parallel}
	In this work we were interested to a particular type of task parallelism:
	\textbf{\textit{Stream parallelism}}.
	
	This means that our tasks are elements of an input stream, of which we don't know a priori the length or the emission rate.\\
	Once the stream elements are available, parallel workers will make independent computations over them and, finally, the manipulated elements will become the output stream.\\
	We recall as main stream parallel patterns \textbf{\textit{Farm}} and \textit{Pipeline}, the former is the object of this work.\\
	
	The \textbf{Farm parallel pattern} is used to model embarrassingly parallel computations. \\The only functional parameter of a farm is the function \(f\) needed to compute the single task.
	Given a stream of input tasks \cite{spm}
	\begin{center}
		\(x_m , . . . , x_1\)\\
	\end{center}
	the farm with function \(f\) computes the output stream as
	\begin{center}
		\(f ( x_m ), . . . , f ( x_1 )\)
	\end{center}
	%Its parallel semantics ensures it will process the single task in a time close to the time needed to compute \(f\) sequentially. The time between the delivery of two different task results, instead, can be made close to the time spent to compute f sequentially divided by the number of parallel agents used to execute the farm, i.e. its parallelism degree 52 .
			
	It's not difficult to see that Farm pattern is really similar to a data parallel problem (in this case a \textit{Map Pattern}). The key difference resides in the input/output data type:
	\begin{itemize}			
		\item \textit{data structures} for Data parallel patterns;
		\item \textit{streams of items} for Farm.
	\end{itemize}
	
	This reveal the main problem of this work, that is the \textit{Data Transfer times} between\\ \textit{\textbf{host memory}} (CPU side) and \textit{\textbf{device memory}} (GPU side), and vice versa.\\ We'll show in detail all aspects of this and other minor problems, together with respective solutions, in \hyperref[chap:logic]{Chapter 3}.

\section{Expectations}
	The main expectation was to show that a not suitable problem, such as Farm parallel pattern, could fit in a GPU. 
	In other words we wanted to see that, running on GPU our streaming parallel code, it could take an advantage near the order of the number of \textbf{SMs} (\textit{\textbf{Streaming Multiprocessors}}).\\
	Looking closer at that this expected results, it means that:
	\begin{itemize}
		\item Data transfer time had in some way to be hidden behind Computation time; 
		
		\item The GPU had to achieve a full \textit{Occupancy} \footnote{We'll insist on occupancy topic in \hyperref[chap:logic]{Chapter 3}.}.\\
	\end{itemize}
	Once we could gain these two factors, no matter what kind of feature GPU has, we expected to get a \(Speedup \approx number \: of \: SMs \).
	The reason why we wanted to see such a speedup is all about gaining some advantages with respect to CPU processing:
	\begin{itemize}
		\item We can delegate our streaming problems to the GPU while the CPU can compute other things, this allow the CPU to not being saturated (especially when stream has high throughput or each element require high computation intensity); 
		
		\item We can split the amount of work between CPU and GPU, the best would be to give respective quantities based on completion time \footnote{See \hyperref[sect:cpugpuscheduling]{Section 3.5}};
	 
		
		\item We hopefully want to see a GPU speedup with respect to the CPU, or see the same performances at worst.
	\end{itemize}

	
\section{Results}
Put a summary on results here.\\

\section{Tools}
\label{sect:tools}
	As mentioned in \hyperref[subs:otherApps]{Subsection 1.1.2} we exploited NVIDIA's CUDA Toolkit. 
	\footnote{In \hyperref[chap:tools]{Chapter 2} will be shown all features and details about the aforementioned tools.}	
	In particular:
	\begin{itemize}
		\item The code was implemented in \texttt{CUDA C++} language, so the compiler was \texttt{nvcc};	
			
		\item The profiling of GPU code performances was supported by \texttt{nvprof} and by its advanced visual version \texttt{NVIDIA Nsight};
				
		\item The debugging was made by using \texttt{cuda-gdb};
				
		\item Studies on GPU Occupancy have been done with \textit{CUDA Occupancy Calculator spreadsheet} and \textit{Occupancy APIs}.
	\end{itemize}
	Tests on the code were implemented as bash scripts and they've been run on two machines:	
	\begin{itemize}
		\item The first with four NVIDIA GPUs \textbf{Tesla P100-PCIE-16GB};
		
		\item The second with four NVIDIA GPUs \textbf{Tesla M40}.
	\end{itemize}
	The code was developed with the following environments:
	\begin{itemize}
		\item \textit{Visual Studio Code} for CUDA C++, Makefile, bash scripts;
		\item \textit{Gedit} for Python scripts.\\\\
	\end{itemize}
		
In next chapters all notions presented in this introduction will be seen in depth.\\
In Chapter 2 there will be an accurate description of all employed tools and how they were used.\\
We'll enter in the core of this work in Chapter 3, where we'll see the logic of the project, with both written and graphical illustrations. In other words here we point out the main ideas and concerns behind our approach and solutions.\\
In Chapter 4 will be presented and explained main implementation choices and there will be listed some fundamental part of the code.\\ 
Then in Chapter 5 will be shown either how experiments and test were set, obtained results and some respective plots.\\
And we'll end up with Conclusions and some final remarks. 
% chapter intro (end)