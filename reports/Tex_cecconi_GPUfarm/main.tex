%\documentclass[11pt, twoside]{report}
%\documentclass[14pt]{extreport}
\documentclass[12pt]{report}
%\usepackage[T1]{fontenc}
\usepackage{ragged2e}
\usepackage{fixltx2e}
\usepackage[utf8]{inputenc}
\usepackage[english]{babel}
\usepackage{listings}
\usepackage{multirow}
\usepackage{wrapfig}

\newcommand\tab[1][1cm]{\hspace*{#1}}
%\usepackage{rotating}
%\usepackage{biblatex}
%\usepackage{helvet}
%\usepackage[a4paper,width=180mm,top=20mm,bottom=20mm,bindingoffset=6mm]{geometry}
\usepackage[framemethod=default]{mdframed}
\usepackage{lipsum,titlesec,fancyhdr,array,multicol,float,graphicx,wrapfig}
%\usepackage{fontspec,todonotes,enumitem,comment,capt-of,amsmath,booktabs,dirtree,minted,tikz,xcolor}
\usepackage[colorlinks]{hyperref}
%\usepackage[font=scriptsize]{caption}
\usepackage[font={small}]{caption}
\usepackage[Sonny]{fncychap}
%\usepackage[sorting=none]{biblatex}
\usepackage[toc,page]{appendix}
\usepackage[left=3cm, right=3cm, top=3cm]{geometry}
\usepackage{graphicx}
\usepackage{subfig}
\usepackage{makecell}

\usepackage{float}

\usepackage{mathtools}

\usepackage{listings,lstautogobble}
\usepackage{xcolor}

\definecolor{RoyalBlue}{cmyk}{1, 0.50, 0, 0}

\lstset{language=C++,
	keywordstyle=\color{RoyalBlue},
	basicstyle=\scriptsize\ttfamily,
	commentstyle=\ttfamily\itshape\color{gray},
	stringstyle=\ttfamily,
	showstringspaces=false,
	breaklines=true,
	frameround=ffff,
	frame=single,
	rulecolor=\color{lightgray},
	autogobble=true
	postbreak=\mbox{\textcolor{blue}{$\hookrightarrow$}\space},
}

%\usepackage[formats]{listings}

%\lstdefineformat{C++}
%{
%	\{=\newline\string\newline\indent,%
%	\}=\newline\noindent\string\newline,%
%	;=[\ ]\string\space,%
%}




\graphicspath{/images}

\renewcommand{\baselinestretch}{1.5} 

\begin{document}
\renewcommand{\familydefault}{\sfdefault}
%\renewcommand{\chap\usepackage[english]{babel}tername}{}
%\renewcommand\thechapter{\Roman{chapter}}

\hypersetup {
    colorlinks,
    citecolor= green,
    filecolor= black,
    linkcolor= black,
    urlcolor= black
}

%\usetikzlibrary{arrows.meta}
%\addbibresource{references.bib}
%\usemintedstyle{bw}

%\definecolor{LightGray}{HTML}{F0F0F0}
	
    %\documentclass{paper}

	%\begin{document}
\begin{titlepage}
	\begin{center}	
		\includegraphics{images/logo_unipi}
		
		\vspace*{1cm}
		\normalsize{					
			Computer Science Department\\
			\vspace{0.5cm}		
			Thesis for Master Degree in Computer Science
		}
		\vspace*{2cm}
		
		
		\huge{ \textbf{GP-GPU: \\  From 	 Data Parallelism \\ to Stream  Parallelism} }
		
		%\vspace*{0.5cm}	
		
		%\Large{From  Data Parallel to Streaming Parallel in GPU computing}
		
		\vspace{1.5cm}
	\end{center}
	
	\vfill
	\vspace*{2cm}
	
	\begin{flushright}
		\Large{	
			Candidato: \textbf{Maria Chiara Cecconi}
			
			\vspace{0.5cm}
			
			Relatore: \textbf{Marco Danelutto}
		}
	\end{flushright}				
\end{titlepage}
%	\end{document}
    \pagenumbering{Roman}
    \tableofcontents
    
\chapter{Introduction}
\pagenumbering{arabic}
\label{chap:intro}
In a scenario where image processing needed to get more and more sophisticated, we saw \textit{graphic processors} follow the change getting increasingly powerful, not only in computation speed but also in flexibility.

The new elasticity provided by \textbf{GPU}s made possible to exploit their benefits for a wide range of non-image-processing problems. This is the beginning of \textbf{GP-GPUs} era.

Despite this, enthusiasm slowed down when scientific community had to deal with problems that seemed to be unsuitable for GP-GPUs. However, several studies and researches showed some good results and possibilities, giving oxygen to keep trying mapping to GPU apparently unsuitable problems. This is the core of this work too.

\pagenumbering{arabic}
\section{Goals}
	The main goal of this thesis is to study the behavior of GPUs when used for different purposes, with respect to the usual ones.
	In particular, we want to use a GPU to run code that models after a \textit{\textbf{stream parallel pattern}}, to understand if we can exploit \textbf{\textit{Streaming Multiprocessors}} (\textbf{SMs}) in parallel and what is the relative efficiency.\\
	This means we investigated the possibility to perform stream parallelism among "small" data parallel tasks, exploiting the different Streaming Multiprocessors, available on the GP-GPU, as workers. \\
	Then we observed ongoings, in terms of \textit{completion time} and \textit{speedup}.\\
	The latter considered as sequential version the case without using CUDA Streams, while the parallel version using them.
	From these speedups we expected to see an improvement slightly below the number of used CUDA Streams, giving us an assessment on the number of employed Streaming Multiprocessors.\\
	The different experiments\footnote{Experiments will be presented in detail in Chapter \ref{chap:experim}.}, gave us the results that we expected. In particular we observed that, using CUDA streams, is possible to achieve a gain proportional to the number of SMs available on a GPU. We could see this behavior especially in arithmetic-intensive applications\footnote{We'll show in Chapter \ref{chap:impl} the different considered and implemented applications in this work. Chapter \ref{chap:experim} will show all results and speedups relative to the different study cases.}.
	
	In next sections, we're going to show some preliminary details about the concepts we've just introduced.

\subsection{GPU Architecture and Data Parallelism}
	\textbf{GPU} (\textbf{\textit{Graphics Processing Unit}}) is a co-processor, generally known as a highly parallel multiprocessor optimized for parallel graphics computing.\\
	Compared with multicore CPUs, manycore GPUs have different architectural design points, one focused on executing many parallel threads efficiently on many cores.
	This is achieved using simpler cores and optimizing for data parallel behavior among groups of threads\cite{pattersonhennessy}.
	
	In most of situations, visual processing can be associated to a \textbf{\textit{data parallel pattern}}.\\
	In general, we can roughly think to an image as a given and known amount of \textit{independent} data upon which we want to do the same computations, over the whole collection. In other words, generally, once the proper granularity of the problem has been chosen, a certain work should be done for each portion of the image.\\
	Considering the above scenario and given that generally a GPU should have to process huge amount of data, we wish to have a lot of threads (lot of cores consequently) doing "the same things" on all data portions.
	
	And that's why GPUs performs their best on data parallel problems. 

\subsection{Other Applications}
\label{subs:otherApps}
	As we've just mentioned, GPUs have been always known to perform at their best on data parallel problems.\\
	However, in recent years, we're moving to \textbf{GP-GPUs} (\textbf{\textit{General-Purpose computing on Graphics Processing Units}}).
	In other words, lately GPUs have been used for applications other than graphics processing.
	
	
	One of the first attempts of executing non-graphical computations on a GPU was a matrix-matrix multiply. In 2001, low-end graphics cards had no floating-point support; floating-point color buffers arrived in 2003.\\
	For the scientific community the addition of floating point meant no more problems on fixed-point arithmetic overflow.\\	
	Other computational advances were possible thanks to programmable shaders\footnote{For example LU factorization with partial pivoting on a GPU was one of the first common kernels, that ran faster than an optimized CPU implementation.}, that broke the rigidity of the \textit{fixed graphics pipeline}.\\
	
	A \textit{\textbf{Graphic Pipeline}} (or rendering pipeline) is a conceptual model that describes what steps a graphics system needs to perform to render a 3D scene to a 2D screen\cite{pipemicrosoft}.\\
	
	Pipeline is generally defined on two levels, i.e. software API level and hardware implementation level. However, it's possible to conceptually define three main stages\footnote{So, given that a graphic pipeline consists of several consequential stages, the service time is given by the slowest stage.}:
	\begin{itemize}
		\item user input, developer controls what the software should execute, passing geometry primitives to next stage;
		\item geometry processing, it executes per-primitive operations, in this phase we've dense computing;
		\item scan conversion phases, it handles per-pixel operations.
	\end{itemize} 
	Only the first stage was programmable in early GPUs, that's why they're often referred to as \textit{fixed-function graphics pipelines}.\\
	Modern graphics hardware, instead, provides developers the ability to customize features in the two last stages\footnote{Generally this is achieved with two techniques: \textit{vertex shader} and \textit{fragment shader}}, with this growth in flexibility started the \textit{programmable pipeline} era\cite{rendering}.\\
	
	We can find pipeline concepts in NVIDIA GPUs too. In fact, each SM of a CUDA device, provides numerous hardware units that are specialized in performing specific task\footnote{For example, texture units provide the ability to execute texture fetches and perform texture filtering, load/Store units fetch and save data to memory etc.}. At the chip level those units provide execution pipelines to which the warp schedulers dispatch instructions\cite{cudapipe}.\\
	% Understanding the utilization of those pipelines and knowing how close they are to the peak performance of the target device are key information for analyzing the efficiency of executing a kernel; and also allows to identify performance bottlenecks caused by oversubscribing to a certain type of pipeline. 
	Now we'll focus on the NVIDIA parallel computing platform used in this thesis, i.e. CUDA\footnote{We'll show some further informations about CUDA in Section \ref{sect:tools}.}.\\
	The introduction of \textbf{NVIDIA}'s \textbf{CUDA} (\textbf{\textit{Compute Unified Device Architecture}}) in 2007, ushered a new era of improved performance for many applications as programming GPUs became simpler: terms such as texels, fragments, and pixels were superseded with \textit{threads}, \textit{vector processing}, \textit{data caches} and \textit{shared memory} \cite{fromCUtoOCL}. \\
	
	From the very beginning of GP-GPUs, scientific general purpose applications on GPUs started from matrix (or vector) computations, that mainly could be referred to as \textbf{\textit{Data parallel}} problems.
	But over time scientific community felt the need to cover other applications, that not necessarily fit data parallel model.\\	
	In particular some of latest researches are moving towards \textbf{\textit{Task parallel}} applications (sometimes also known as \textit{Irregular-Workloads parallel patterns})\cite{backtrack}.\\
	
	An example of non-data parallel problem is the \textit{backtracking paradigm}.
	It's often at the core of compute-and-memory-intensive problems and we can find its application in different cases, such as: constraint satisfaction in AI, maximal clique enumeration in graph mining and k-d tree traversal for ray tracing in graphics.
	
	Some computational parallel patterns perform effectively on a GPU, while the effectiveness of others is still an open issue.\\
	For example, in several studies it was highlighted that memory-bound algorithms on the GPU perform at the same level or worse than the corresponding CPU implementation. 
	
	Task-parallel systems must deal with different situations, with respect to those present in data parallelism, e.g.:
	\begin{itemize}
		\item Handle divergent workflows;
		\item Handle irregular parallelism;
		\item Respect dependencies between tasks;
		\item Implementing efficient load balancing.
	\end{itemize}
	
	Those requirements can lead to inefficient use of the GPU memory hierarchy and SIMD-optimized\footnote{We'll see SIMD architecture, CUDA memory hierarchy and other architectural details in Section \ref{sect:nvidiaarch}.} GPU multi-processors.
	
	However, there have been backtracking-based or other task-parallel algorithms successfully mapped onto the GPU: the most visible example is in \textit{ray tracing} rendering technique; other examples are \textit{H.264 Intra Prediction} video compression encoding, \textit{Reyes Rendering} and Deferred Lighting.
		
	However, in general we cannot expect an order of magnitude increase in performance. Rather, a more realistic goal is to perform at one-two times the CPU performance, which opens up the possibility of building future non-data-parallel algorithms on heterogeneous hardware (such as CPU-GPU clusters) and performing workload-based optimizations \cite{backtrack}.
 

\subsection{GP-GPUs and Stream Parallel}
\label{subs:gpgpustreampar}
	In this work we were interested to a particular type of task parallelism:\\
	\textbf{\textit{Stream parallelism}}.
	
	This means that our tasks are elements from an input stream, of which we don't know a priori the length or the interval times between tasks.\\
	Once the stream elements are available, parallel workers will make independent computations over them and, finally, the manipulated elements should be delivered to some output stream.\\
	We recall as main stream parallel patterns the \textit{Farm} and the \textit{Pipeline}, the former being the main subject in this thesis.\\
	
	The \textbf{Farm parallel pattern}\footnote{The Farm parallel pattern will be seen in detail in Chapter \ref{chap:logic}.} is used to model embarrassingly parallel computations. 	
	This pattern computes in parallel the same function \(f:\alpha\rightarrow\beta\) over all the items appearing in the input stream of type \(\alpha\) \texttt{stream} delivering the results on the output stream of type \(\beta\) \texttt{stream}.\\
	The model of computation of the task-farm pattern consists of three logical entities: the \textit{Emitter}, that is in charge of accepting input data streams and to assign the data to the Workers; a pool of \textit{Workers} which compute the function \textit{f} in  parallel over different stream elements; the \textit{Collector} that non-deterministically gathers Workers' partial results and eventually produces the final result.
	
	The Emitter, the set of Workers and the Collector interact in a pipeline way using a data-flow model which can be implemented in several different ways depending on the target platform.  For example, the Emitter and Collector, could be implemented in a centralized way using a single thread, or in a partially or fully distributed way.\\
	Farm's workers can be formed by any other pattern\cite{spm}.
	%An  interesting  result  concerning  composition  ofpipeandfarmpatterns is the following [16]:pipe(seq(f1), seq(f2))≡farm(seqcomp(f1,f2), n)wherenis a non-functional parameter representing the num-ber  of  Workers  in  thefarmpattern.   
	In the general case, input/output data ordering may be altered due to the different relative speeds of the workers executing the distinct stream items. If ordering is important, it can be enforced by the Collector or by the scheduling/gathering policies of the farm pattern.  
	%We callofarmthe instance of thefarmpattern that preserves input/output ordering.
	
	\textit{Master-Worker} is a specialization of the task-farm pattern where the Emitter and Collector are collapsed in a single entity (called \textit{master}).\\
	The Workers deliver computed results back to the master. The master schedules received input tasks toward the pool of workers trying to balance their workload\cite{parpattbench}.\\
	
	In the case of this thesis, we exploited \textit{Streaming Multiprocessors as Farm Workers}, each computing one or more kernel executions, so the function \(f\), in our case, is given by the kernel code. Furthermore we assumed that Emitter and Collector were both managed by CPU-side.\\
	This means that we considered \textit{GPU as Worker} and \textit{CPU as Master}, in particular our implementation is roughly organized as follows:
	\begin{itemize}
		\item \textit{\textbf{host-side} code} (code executed by CPU) manages input stream and forwards items to GPU according to a  \textit{Round-Robin} scheduling policy, furthermore host manages results arriving from the GPU;
		\item \textit{\textbf{device-side} code} (code executed by GPU) mainly executes the worker function \textit{f}, here called \textit{kernel}\footnote{In Chapter \ref{chap:tools}, the concept of kernels will be explained, together with other main features from CUDA C/C++ language.}.
	\end{itemize}
	Since several kernel calls are executed in parallel by a certain SM, we assumed Streaming Multiprocessors to be the workers.\\
	
	For completeness, we'll also briefly introduce \textit{data parallelism}.\\
	It refers to those problems where more workers perform the same task on different portion of data.
	Generally this is achieved having different parallel entities (e.g. threads), such that they execute the same code on different parts of the input data structure\cite{parpattbench}.
	
	One of the most important data parallel pattern, is \textit{Map}. It computes a given function \(f:\alpha\rightarrow\beta\) over all the  data items of an input collection whose elements have type \(\alpha\). The  produced output is a collection of items of type \(\beta\).  Given the input collection \(x_{1},x_{2},...,x_{N}\), the output collection is \(y_{1},y_{2},...,y_{N}\) where \(y_{i}=f(xi)\) for \(i=1,2,...,N\).\\
	Here each data item in the input collection is independent from the other items, so all the elements can be computed in parallel \cite{parpattbench,spm}.\\
	
	The model of computation of the map pattern is very similar to the one described for the farm pattern.\\
	The key difference is in the input/output data type:
	\begin{itemize}			
		\item \textit{data structures} for Map;
		\item \textit{streams of items} for Farm.
	\end{itemize}
	So, the farm pattern works on streams of independent data (a stream may be unbounded), while the map pattern receives a data collection, of a fixed number of items, that is partitioned among the available computing resources\cite{spm}.\\
	In other words, the general difference between data parallel and stream parallel patterns, is that in the latter neither the full sequence nor the number of items in the sequence are known in advance\cite{streamparpatt}.

	The above difference points out one of the main problems of this work: the \textit{Data Transfer times} between  \textit{\textbf{host memory}} (CPU side) and \textit{\textbf{device memory}} (GPU side), and vice versa.\\ 
	In particular, host/device memory copies  overhead is a problem because:
	\begin{itemize}
		\item it represents an amount of time spent in memory operations, instead of necessary computations;
		\item it introduces host/device synchronizations\footnote{This holds for the synchronous version of the command for host/device data transfers \texttt{cudaMemcpy}.}, for example GPU waits for input data copy to end and CPU waits for output data to be fully copied back\footnote{We'll see in Chapters \ref{chap:tools}-\ref{chap:logic} that we'll try to hide this overload by using asynchronous calls.}.
	\end{itemize}
	So data transfers, may represent a bottleneck, especially with respect to the arrival rate of input stream items.
	
	We'll show in detail all aspects of this and other problems, together with respective solutions, in \hyperref[chap:logic]{Chapter 3}.

\section{Expectations}
	The main expectation was to show that a not suitable problem, such as Farm parallel pattern, could fit in a GPU architecture.\\
	It's important to point out that, in particular, we're modeling a Streaming parallel pattern having small data parallel portions as tasks.\\
	In other words, running on GPU our stream parallel code, it calls a stream of light kernels, each computing one of the small data parallel task from the Farm input stream.\\
	Then, we wanted to see that in this way we could take an advantage in the order of the \textbf{SMs}' number, available in the target GPU.\\

	Looking closer at that this expected results, it means that:
	\begin{itemize}
		\item Data transfer time has to be hidden, in some way, by computation time;
		
		\item Kernel executions should take enough time in computations, in order to have chances to achieve overlapping between different kernel executions;
		
		\item The GPU needs to achieve a good \textit{Occupancy} \footnote{We'll insist on occupancy topic in \hyperref[chap:logic]{Chapter 3}.}.
	\end{itemize}
	Once we could achieve these factors, no matter what kind of feature GPU has, we expected to get a \(Speedup \approx number \  of \  SMs \).
	The reason why we wanted to see such a speedup is all about possibly gaining some advantages, with respect to CPU processing:
	\begin{itemize}
		\item We can delegate stream parallel problems to the GPU while the CPU can compute other things, this allows to not saturate the CPU (especially when stream has high throughput or each element requires high computation intensity); 
		
		\item We can split the amount of work between CPU and GPU, the best would be to give respective quantities based on completion time\cite{cpugpumix}; %\footnote{See \hyperref[sect:cpugpuscheduling]{Section 3.5}};
	 
		
		\item We hopefully want to see a GPU speedup with respect to the CPU, or see the same performances at worst.
	\end{itemize}

	
\section{Results}
	At this point, it was useful to experiment different applications, i.e. different kernels codes. More precisely we implemented three types:
	\begin{itemize}
		\item \textbf{Compute-bound}, where the amount of computations, performed by the kernel, is  greater than the amount of memory operations;
		\item \textbf{Memory-bound}, dominated by memory operations;
		\item \textbf{Divergent flows} (and memory bound too), this means we have a kernel with a lot of branching code\footnote{Chapter \ref{chap:experim} explains details about compute-bound, memory-bound and divergent flows in kernels.}.
	\end{itemize}
	In the case of our kernels, memory-bound operations are mainly given by load/store from the Global Memory\footnote{In Chapter \ref{chap:tools} we'll see an overview on the memory organization in GPUs.} to registers and vice versa\cite{cudabestpractices}.
	
	We observed that compute-bound kernel gave us the expected results for GPU Farm, that is gaining a good overlap between kernel executions and, so, a speedup proportional to SMs number.
	While memory-bound gave a really low amount of overlap and speedup and for the divergent flow case we've even worse performances, given that we had no speedup at all.\\
	Anyway, the above results represent what we expected and we'll give all relative details in Chapter \ref{chap:experim}.
	
	
\section{Tools}
\label{sect:tools}
	As mentioned in Subsection \ref{subs:otherApps} we mainly exploited NVIDIA's CUDA Toolkit\footnote{\hyperref[chap:tools]{Chapter 2} will show features and details about CUDA Toolkit and all other tools that have been used.}.
	In particular:
	\begin{itemize}
		\item The code was implemented in \texttt{CUDA C++} language, so the compiler was \texttt{nvcc};	
			
		\item The profiling of GPU code performances was supported by \texttt{nvprof} and by its advanced visual version \texttt{NVIDIA Visual Profiler};
				
		\item The debugging was made by using \texttt{cuda-gdb};
				
		\item Studies on GPU Occupancy have been done with \textit{CUDA Occupancy Calculator spreadsheet} and \textit{Occupancy APIs}.
	\end{itemize}
	Tests on the code were implemented as bash scripts and they've been run on two machines:	
	\begin{itemize}
		\item The first with four NVIDIA GPUs \textbf{Tesla P100-PCIE-16GB};
		
		\item The second with four NVIDIA GPUs \textbf{Tesla M40}.
	\end{itemize}
	The code was developed with the following environments:
	\begin{itemize}
		\item \textit{Visual Studio Code} for CUDA C++, Makefile, bash scripts;
		\item \textit{Gedit} for Python scripts.\\
	\end{itemize}
		
In next chapters all the concepts briefly outlined in this introduction will be discussed in depth.\\
Chapter 2 introduces an accurate description of all employed tools and how they were used.\\
Then Chapter 3 explains the logic of the project, with both text and graphical illustrations. In other words here we point out the main ideas and concerns, together with relative solutions, behind our approach.\\
Chapter 4 presents and explains main implementation choices and there will be listed some fundamental part of the code.\\ 
Then Chapter 5 shows how both experiments and tests are set, then the obtained results and some respective plots will be presented.\\
Finally, conclusions give an overall view of the thesis, some final remarks and future works suggestions. 
% chapter intro (end)
    %\addcontentsline{toc}{section}{Unnumbered Section}
\chapter{Tools} \label{chap:tools}

In this project all tools and elaborations were made in GNU/Linux environment.
We had two available remote computers, to which we connect by \texttt{ssh} command on terminal.\\
In particular we worked on the following machines:
	\begin{enumerate}
		\item \textbf{Local host}
		\begin{itemize} 
			\item Ubuntu 14.04 LTS, (4.4.0-148-generic x86\_64)
			\item 1 CPU Intel® Core™ i5 CPU M 450 @ 2.40GHz x 4 
		\end{itemize}
		
		\item \textbf{P100 remote server}
		\begin{itemize}
			\item Ubuntu 18.04.2 LTS (4.15.0-43-generic x86\_64)	
			\item 80 CPUs Intel(R) Xeon(R) CPU E5-2698 v4 @ 2.20GHz		
			\item 4 GPUs Tesla P100-PCIE-16GB\\\\
		\end{itemize}
		 
		\item\textbf{ M40 remote server}
		\begin{itemize}
			\item Ubuntu 16.04.6 LTS (4.4.0-154-generic x86\_64)
			\item 48 CPUs Intel(R) Xeon(R) CPU E5-2670 v3 @ 2.30GHz
			\item 4 GPUs NVIDIA Tesla M40\\
		\end{itemize}
	\end{enumerate}
	Given that this work is focused on the use of the remote GPUs, their main specifics are listed in Table \ref{tab:gpuspecs}. All the following informations have been get by executing \texttt{cudaDeviceQuery} application (located inside samples of CUDA Toolkit).\\
	\begin{table}	
	\begin{tabular}{|c | c c |} 
		\hline
  & \textbf{Tesla P100} & \textbf{Tesla M40} \\ [0.5ex] 
		\hline\hline
		
		\textbf{Driver/Runtime Version} & 10.1  & 10.1 \\ 
		\hline
		
		\textbf{CUDA Capability} & 6.0 & 5.2 \\
		\hline
		\textbf{\makecell{Tot. \\global memory amount}} & 16281 MBytes & 11449 MBytes \\
		\hline
		
		\textbf{Multiprocessors} & 56 & 24 \\
		\hline
		
		\textbf{\makecell{CUDA Cores/MP \\(Tot. CUDA cores)}} & 64 (3584) & 128 (3072) \\ %[1ex] 
		\hline
		
		\textbf{GPU Max Clock rate} & 1329 MHz (1.33 GHz) & 1112 MHz (1.11 GHz) \\ 
		\hline
		
		\textbf{\makecell{Tot. amount\\ constant memory} } & 65536 bytes & 65536 bytes \\ 
		\hline
		
		\textbf{\makecell{Tot. amount\\ shared memory/block}} & 49152 bytes & 49152 bytes \\ 
		\hline
		
		\textbf{\makecell{Tot.\\ \#registers available/block}} & 65536 & 65536 \\ 
		\hline
		
		\textbf{Warp size} & 32 & 32\\
		\hline
		
		\textbf{\makecell{Maximum\\ \#threads/multiprocessor}} & 2048 & 2048 \\
		\hline
		
		\textbf{Max \#threads/block} & 1024 & 1024 \\
		\hline
		
		\textbf{\makecell{Max thread block dimensions \\(x,y,z)}} & (1024, 1024, 64) & (1024, 1024, 64) \\
		\hline 
		
		\textbf{\makecell{Max grid size dimensions\\ (x,y,z)}} & (2147483647, 65535, 65535) & (2147483647, 65535, 65535) \\
		\hline
		
		 \textbf{\makecell{Concurrent copy \& \\ kernel exec}} & Yes with 2 copy engine(s) & Yes with 2 copy engine(s) \\
		\hline		
	\end{tabular}
	\caption{GPUs specifics for the two remote machines employed in this project.}	
	\label{tab:gpuspecs}		
	\end{table}
%	Here we present most important specifics for \textbf{P100}:
%	\begin{itemize}
%		\item Device: "Tesla P100-PCIE-16GB"
%		\item CUDA Driver/Runtime Version: 10.1 / 10.1
%		\item CUDA Capability version number: 6.0
%		\item Total amount of global memory: 16281 MBytes (17071734784 bytes)
%		\item (56) Multiprocessors, ( 64) CUDA Cores/MP: 3584 CUDA Cores
%		\item GPU Max Clock rate: 1329 MHz (1.33 GHz)
%		%Memory Clock rate:                             715 Mhz
%		%Memory Bus Width:                              4096-bit
%		%L2 Cache Size:                                 4194304 bytes
%		%Maximum Texture Dimension Size (x,y,z)         1D=(131072), 2D=(131072, 65536), 3D=(16384, 16384, 16384)
%		%Maximum Layered 1D Texture Size, (num) layers  1D=(32768), 2048 layers
%		%Maximum Layered 2D Texture Size, (num) layers  2D=(32768, 32768), 2048 layers
%		\item Total amount of constant memory: 65536 bytes
%		\item Total amount of shared memory per block: 49152 bytes
%		\item Total number of registers available per block: 65536
%		\item Warp size: 32
%		\item Maximum number of threads per multiprocessor:  2048
%		\item Maximum number of threads per block: 1024
%		\item Max dimension size of a thread block (x,y,z): (1024, 1024, 64)
%		\item Max dimension size of a grid size    (x,y,z): (2147483647, 65535, 65535)
%		%Maximum memory pitch:                          2147483647 bytes
%		%Texture alignment:                             512 bytes
%		\item Concurrent copy and kernel execution:          Yes with 2 copy engine(s)\\
%		%Run time limit on kernels:                     No
%		%Integrated GPU sharing Host Memory:            No
%		%Support host page-locked memory mapping:       Yes
%		%Alignment requirement for Surfaces:            Yes
%		%Device has ECC support:                        Enabled
%		%Device supports Unified Addressing (UVA):      Yes
%		%Device supports Compute Preemption:            Yes
%		%Supports Cooperative Kernel Launch:            Yes
%		%Supports MultiDevice Co-op Kernel Launch:      Yes
%		%Device PCI Domain ID / Bus ID / location ID:   0 / 130 / 0
%	\end{itemize}
%	
%	Then we present specifics for \textbf{M40}:
%	\begin{itemize}
%		\item Device: "Tesla M40"
%		\item CUDA Driver Version / Runtime Version          10.1 / 10.1
%		\item CUDA Capability Major/Minor version number:    5.2
%		\item Total amount of global memory:                 11449 MBytes (12004753408 bytes)
%		\item (24) Multiprocessors, (128) CUDA Cores/MP:     3072 CUDA Cores
%		GPU Max Clock rate:                            1112 MHz (1.11 GHz)
%		%Memory Clock rate:                             3004 Mhz
%		%Memory Bus Width:                              384-bit
%		%L2 Cache Size:                                 3145728 bytes
%		%Maximum Texture Dimension Size (x,y,z)         1D=(65536), 2D=(65536, 65536), 3D=(4096, 4096, 4096)
%		%Maximum Layered 1D Texture Size, (num) layers  1D=(16384), 2048 layers
%		%Maximum Layered 2D Texture Size, (num) layers  2D=(16384, 16384), 2048 layers
%		\item Total amount of constant memory:               65536 bytes
%		\item Total amount of shared memory per block:       49152 bytes
%		\item Total number of registers available per block: 65536
%		\item Warp size:                                     32
%		\item Maximum number of threads per multiprocessor:  2048
%		\item Maximum number of threads per block:           1024
%		\item Max dimension size of a thread block (x,y,z): (1024, 1024, 64)
%		\item Max dimension size of a grid size    (x,y,z): (2147483647, 65535, 65535)
%		%Maximum memory pitch:                          2147483647 bytes
%		%Texture alignment:                             512 bytes
%		\item Concurrent copy and kernel execution:          Yes with 2 copy engine(s)\\
%		%Run time limit on kernels:                     No
%		%Integrated GPU sharing Host Memory:            No
%		%Support host page-locked memory mapping:       Yes
%		%Alignment requirement for Surfaces:            Yes
%		%Device has ECC support:                        Enabled
%		%Device supports Unified Addressing (UVA):      Yes
%		%Device supports Compute Preemption:            No
%		%Supports Cooperative Kernel Launch:            No
%		%Supports MultiDevice Co-op Kernel Launch:      No
%		%Device PCI Domain ID / Bus ID / location ID:   0 / 130 / 0	
%	\end{itemize}
In this work we mainly made use of many tools in the CUDA Toolkit. In the following section will be presented all of employed stuff, with some specifications and how they've been exploited during this project.

\section{CUDA}
	In November 2006, NVIDIA introduced CUDA, a general purpose parallel computing platform and programming model provided for compute engine in NVIDIA GPUs, to solve many complex computational problems (sometimes in a more efficient way than on a CPU).\\
	%CUDA comes with a software environment that allows developers to use C as a high-level programming language. \\ 
	
	The advent of multicore CPUs and manycore GPUs means that mainstream processor chips are now parallel systems and their parallelism continues to scale with Moore's law.\\
	%The CUDA challenge is to develop application software that scales its parallelism to leverage the increasing number of processor cores, while maintaining a low learning curve for programmers familiar with standard programming languages such as C.\\
	We used CUDA with C++ support.
	At its core are three key abstractions that are exposed to the programmer as a minimal
	set of language extensions:
	\begin{itemize}
		\item A hierarchy of thread groups;
		
		\item Shared memories;
		
		\item Barrier synchronization.
	\end{itemize} 
	
	These abstractions provide fine-grained data parallelism and thread parallelism, nested within coarse-grained data parallelism and task parallelism.\\
	This makes possible to partition the problem into coarse sub-problems \textendash solved independently in parallel by \textit{blocks} of threads \textendash, and each sub-problem into finer pieces \textendash solved cooperatively in parallel by all \textit{threads} within the block \textendash.
	
	\begin{figure}[H]
		%\vspace*{-2.4cm}
		\centering
		\includegraphics[width=0.8\textwidth]{images/cudaSMs.png}
		\caption{GPU scalability.}
		\label{fig:cudaSM}
	\end{figure}
	Indeed, \textit{each block of threads can be scheduled on any of the available SMs within a GPU, in any order, concurrently or sequentially}, so that a compiled CUDA program can execute on any number of multiprocessors as illustrated by Figure \ref{fig:cudaSM}, and only the runtime system needs to know the physical multiprocessor count.
	This programming model scales on the number of multiprocessors and memory partitions \cite{cudaguide}.  

\section{Profilers}	
	NVIDIA profiling tools were useful to optimize in performance our CUDA applications, we used three different versions: the \texttt{nvprof} profiling tool enables you to collect and view profiling data from the command-line. 
	
	%Note that Visual Profiler and nvprof will be deprecated in a future CUDA release. It is recommended to use next-generation tools NVIDIA Nsight Compute for GPU profiling and NVIDIA Nsight Systems for GPU and CPU sampling and tracing.
	
	%NVIDIA Nsight Compute is an interactive kernel profiler for CUDA applications. It provides detailed performance metrics and API debugging via a user interface and command line tool. In addition, its baseline feature allows users to compare results within the tool. Nsight Compute provides a customizable and data-driven user interface and metric collection and can be extended with analysis scripts for post-processing results.
	
	%NVIDIA Nsight Systems is a system-wide performance analysis tool designed to visualize an application’s algorithms, help you identify the largest opportunities to optimize, and tune to scale efficiently across any quantity or size of CPUs and GPUs; from large server to our smallest SoC.
	
	\subsection{nvprof}
%	CUDA Pro Tip: nvprof is Your Handy Universal GPU Profiler
%	By Mark Harris | October 23, 2013 Tags: CUDA, Development Tools and Libraries, GPU, Pro Tip, Profiling, Programming Languages and Compilers, Python, Tools
	\texttt{nvprof} was added to CUDA Toolkit with CUDA 5. It is a command-line profiler, so it's a GUI-less version of the graphical profiling features available in the NVIDIA Visual Profiler.\\
	The \texttt{nvprof} profiler enables the collection of a timeline of CUDA-related activities on both CPU and GPU, including kernel execution, memory transfers, memory set and CUDA API calls and events or metrics for CUDA kernels.\\
	After all data is collected, profiling results are displayed in the console \footnote{The textual output of the profiler is redirected to \texttt{stderr} by default.} or can be saved in a log file for later viewing \footnote{Or for later import into either \texttt{nvprof} or the \texttt{NVIDIA Visual Profiler)}.} \cite{profilersguide, nvprofarticle}.\\
	\texttt{nvprof} operates in different modes: \textit{Summary Mode}, \textit{GPU-Trace and API-Trace Modes}, \textit{Event/metric Summary Mode} and  \textit{Event/metric Trace Mode}.\\	
	For our purposes we used only \textit{Summary Mode}, this is the default operating mode, where we have a single result line for each kernel function and each type of CUDA memory copy performed by the application (for each operation type are shown number of calls and total, max, min and average time) \cite{profilersguide}.
	
	We used it, in some situations, as a quick check, for example we exploited it to see if the application wasn't running kernels on the GPU at all, or it was performing an unexpected number of memory copies, etc.
	To this aim it's enough to run the application with\\
	\texttt{nvprof ./myApp arg0 arg1 ...}\\
	Although when we launched our tests, we wanted to consult profiling results after running or whenever was necessary, it was useful especially when we had to compare them with time probes inside the code. So we used \texttt{--log-file} option to redirect the output to files for deferred examination. \\
	\texttt{nvprof} revealed peculiarly suitable for remote profiling. That's because of the fact command line is faster to check and save an application profiling.
	  
	\subsection{NVIDIA Visual Profiler}
	The NVIDIA Visual Profiler, introduced in 2008, is a performance profiling tool providing visual feedback for optimizing CUDA C/C++ applications.
	
	The Visual Profiler displays a timeline of an application's activity on both the CPU and GPU to make performance improvement, it analyzes the application to detect potential bottlenecks, using graphical views, that allow to check memory transfers, kernel launches, and other API functions on the same timeline \cite{profilersguide}.\\
	 
	We used the standalone version of the Visual Profiler, \texttt{nvvp}. 
	Furthermore we used NVIDIA Nsight Systems, the advanced version of Visual Profiler, this is a low overhead performance analysis tool that provides insights to optimize software, to investigate for bottlenecks. It also identifies issues, such as GPU starvation, unnecessary GPU synchronization, insufficient CPU parallelizing, unexpectedly expensive algorithms across the CPUs and GPUs of their target platform. 
	 
	In particular in this project \texttt{Nsight Systems} was used in developing phase, to check if code was properly written to hide as much as possible data transfers \footnote{We'll see in detail the Overlapping topic and how it was managed in \hyperref[chap:logic]{Chapter 3}.}, even though sometimes it may happens that profilers introduce some sampling synchronization, giving not fully reliable visual results.
	

\section{CUDA C/C++}
	Here we'll briefly introduce main concepts behind the CUDA programming model, by outlining how they are exposed in C.
	Especially we'll show important notions about features involved in this project and how/why these were included.
	\begin{wrapfigure}{l}{0.58\textwidth}
		\raggedleft
		
		\includegraphics[width=0.58\textwidth]{images/gridblocks.png}
		\caption{Above: a Grid formed by Blocks.\\ Below: a Block formed by Threads.}
		\label{fig:gridblock}
	\end{wrapfigure}
	
	\subsection{Kernels} 
	\label{subs:ker}
	CUDA C allows to define particular C functions, called \textbf{\textit{kernels}}, when called, these are executed N times in parallel by N different CUDA threads, as	opposed to only once like regular C functions.
	
	A kernel is defined using the \texttt{\_\_global\_\_} declaration specifier. The number of CUDA threads that will execute the kernel for a given call is specified using this special execution configuration syntax: 
	\texttt{<<<...>>>}.\\
	Each thread executing the kernel is given a unique thread ID, accessible within the kernel through the built-in \texttt{threadIdx} variable \cite{cudaguide}.

	\subsection{Thread Hierarchy}  
	In practice \texttt{threadIdx} is a 3-component vector, so that threads can be identified using either one, or two, or three dimensional thread index.
	In turn these threads will form	either one, or two, or three dimensional block of threads, called a
	\textbf{\textit{thread block}}.
	This provides a way to invoke computation across the elements in domains such as a vectors, matrices, or volumes.
	%The index of a thread and its thread ID relate to each other in a straightforward way:
	%\begin{itemize}
	%	\item For a one-dimensional block, they are the same; 
		
	%	\item for a two-dimensional block of size \((D_{x} ,	D_{y}) \space \rightarrow \space\)  a thread of index \((x, y)\) has \(threadID = (x + y \cdot D_{x} );\)
		
	%	\item for a three-dimensional block of size \((D_{x} ,	D_{y} ,	D_{z}) \space \rightarrow \space\)  a thread of index \((x, y, z)\) has \(threadID = (x + y \cdot D_{x} + z \cdot D_{x} \cdot D_{y});\)
	%\end{itemize}
	There is a limit to the number of threads per block, since \textit{all threads of a block are expected to reside on the same processor core and must share the limited memory resources of that core}. On current GPUs, and on the two we worked on, a thread block may contain up to 1024 threads \footnote{see Table \ref{tab:gpuspecs} for limits in the machines we used).}.\\
	However, a kernel can be executed by multiple equally-shaped thread blocks, so that\\
	\(Total number of threads = \#threadsPerBlock \cdot \#blocks\)\\	 
	Blocks in turn are organized into either one, or two, or three dimensional \textit{\textbf{grid of thread blocks}} as illustrated by Figure \ref{fig:gridblock}.	
	So, the number of blocks in a grid is usually dictated by the size of the data being processed or the number of processors in the system.
	The number of \textit{threads per block} and the number of \textit{blocks per grid} specified in the 	\texttt{<<<...>>>} syntax can be of type \texttt{int} or \texttt{dim3}.\\
	The dimension of the thread block, block index and thread index are accessible within the kernel through the respective built-in variables: \texttt{blockDim}, \texttt{blockIdx}, \texttt{threadIdx}. \cite{cudaguide}.

	\subsection{CUDA Streams}
	\label{subs:streams} 
	\textbf{Overlap of Data Transfer and Kernel Execution}\\
	Some devices can perform an asynchronous memory copy to or from the GPU	concurrently with kernel execution. 
	It's possible to query this capability by checking the \texttt{asyncEngineCount} device property \footnote{ See \textit{Device Enumeration} in Table \ref{tab:gpuspecs}, For both of our machines, from the \textit{deviceQuery}, we get 2 copy engines.}, which is greater than zero for devices that support it. If host memory is involved in the copy, it must be
	\textit{\textbf{page-locked}}.\\\\
	%In particular some devices, of compute capability 2.x and higher, can overlap copies to and from the device. Applications may query this capability by checking the \texttt{asyncEngineCount} device property (see Device Enumeration), which is equal to 2 for devices that support	it \footnote{This is the case for both of our machines. \\ See the last line of \texttt{deviceQuery} in Table \ref{tab:gpuspecs}}. 
	\Large \textbf{Streams}\normalsize\\
	Applications manage the concurrent operations described above through \textbf{streams}. A stream is a sequence of commands (possibly issued by different host threads) that execute in order. On the other hand, a stream may execute their commands out of order or concurrently with respect to another; this behavior is not guaranteed and	should not be relied upon for correctness (e.g. inter-kernel communication is undefined) \cite{cudaguide}.\\
	A brief code example:
	\begin{lstlisting}
	cudaStream_t stream[2];
	for (int i = 0; i < 2; ++i)
		cudaStreamCreate(&stream[i]);
	float* hostPtr;
	cudaMallocHost(&hostPtr, 2 * size);
	\end{lstlisting}	

	Each of these streams is defined, by the following code sample, as a sequence of one memory copy \(host \rightarrow device\), one kernel launch, and one memory copy \(host \leftarrow device\):
	\begin{lstlisting}
	for (int i = 0; i < 2; ++i) {
		cudaMemcpyAsync(inputDevPtr + i * size, hostPtr + i * size, size, cudaMemcpyHostToDevice, stream[i]);
		MyKernel <<<100, 512, 0, stream[i]>>> (outputDevPtr + i * size, inputDevPtr + i * size, size);
		cudaMemcpyAsync(hostPtr + i * size, outputDevPtr + i * size, size, cudaMemcpyDeviceToHost, stream[i]);
	}
	\end{lstlisting}
	
	Each stream copies its portion of input array \texttt{hostPtr} to array \texttt{inputDevPtr} in device memory, processes \texttt{inputDevPtr} on the device by calling \texttt{MyKernel()}, and copies
	the result \texttt{outputDevPtr} back to the same portion of \texttt{hostPtr}. Note that \texttt{hostPtr} must point to page-locked host memory for any overlap to
	occur.\\
	Streams are released by calling \texttt{cudaStreamDestroy()}.
	\begin{lstlisting}
	for (int i = 0; i < 2; ++i)
		cudaStreamDestroy(stream[i]);
	\end{lstlisting}
	
%	In case the device is still doing work in the stream when cudaStreamDestroy() is 	called, the function will return immediately and the resources associated with the stream 	will be released automatically once the device has completed all work in the stream.


%	3.2.5.5.2. Default Stream
%	Kernel launches and host <-> device memory copies that do not specify any stream
%	parameter, or equivalently that set the stream parameter to zero, are issued to the default
%	stream. They are therefore executed in order.
%	For code that is compiled using the --default-stream per-thread compilation flag
%	(or that defines the CUDA\_API\_PER\_THREAD\_DEFAULT\_STREAM macro before including
%	CUDA headers ( cuda.h and cuda\_runtime.h )), the default stream is a regular stream
%	and each host thread has its own default stream.
%	For code that is compiled using the --default-stream legacy compilation flag, the
%	default stream is a special stream called the NULL stream and each device has a single
%	NULL stream used for all host threads. The NULL stream is special as it causes implicit
%	synchronization as described in Implicit Synchronization.
%	For code that is compiled without specifying a --default-stream compilation flag, --
%	default-stream legacy is assumed as the default.
	
%	QUESTO LO METTEREI NEL CAPITOLO IMPLEMENTATION
%	3.2.5.5.3. Explicit Synchronization
%	There are various ways to explicitly synchronize streams with each other.
%	cudaDeviceSynchronize() waits until all preceding commands in all streams of all
%	host threads have completed.
%	cudaStreamSynchronize() takes a stream as a parameter and waits until all preceding
%	commands in the given stream have completed. It can be used to synchronize the host
%	with a specific stream, allowing other streams to continue executing on the device.
%	www.nvidia.com
%	CUDA C Programming Guide
%	PG-02829-001\_v10.0 | 33Programming Interface
%	cudaStreamWaitEvent() takes a stream and an event as parameters (see Events for
%	a description of events)and makes all the commands added to the given stream after
%	the call to cudaStreamWaitEvent() delay their execution until the given event has
%	completed. The stream can be 0, in which case all the commands added to any stream
%	after the call to cudaStreamWaitEvent() wait on the event.
%	cudaStreamQuery() provides applications with a way to know if all preceding
%	commands in a stream have completed.
%	To avoid unnecessary slowdowns, all these synchronization functions are usually best
%	used for timing purposes or to isolate a launch or memory copy that is failing.
%	3.2.5.5.4. Implicit Synchronization
%	Two commands from different streams cannot run concurrently if any one of the
%	following operations is issued in-between them by the host thread:
%
%	-a page-locked host memory allocation,
%	-a device memory allocation,
%	-a device memory set,
%	-a memory copy between two addresses to the same device memory,
%	-any CUDA command to the NULL stream,
%	-a switch between the L1/shared memory configurations described in Compute
%	Capability 3.x and Compute Capability 7.x.
%	For devices that support concurrent kernel execution and are of compute capability 3.0
%	or lower, any operation that requires a dependency check to see if a streamed kernel
%	launch is complete:
%
%	-Can start executing only when all thread blocks of all prior kernel launches from any
%	stream in the CUDA context have started executing;
%	-Blocks all later kernel launches from any stream in the CUDA context until the kernel launch being checked is complete.
%	Operations that require a dependency check include any other commands within the
%	same stream as the launch being checked and any call to cudaStreamQuery() on that
%	stream. Therefore, applications should follow these guidelines to improve their potential
%	for concurrent kernel execution:
%	-All independent operations should be issued before dependent operations,
%	-Synchronization of any kind should be delayed as long as possible.
%	3.2.5.5.5. Overlapping Behavior
%	The amount of execution overlap between two streams depends on the order in which
%	the commands are issued to each stream and whether or not the device supports
%	overlap of data transfer and kernel execution (see Overlap of Data Transfer and Kernel
%	Execution), concurrent kernel execution (see Concurrent Kernel Execution), and/or
%	concurrent data transfers (see Concurrent Data Transfers).
%	For example, on devices that do not support concurrent data transfers, the two streams
%	of the code sample of Creation and Destruction do not overlap at all because the
%	www.nvidia.com
%	CUDA C Programming Guide
%	PG-02829-001\_v10.0 | 34Programming Interface
%	memory copy from host to device is issued to stream[1] after the memory copy from
%	device to host is issued to stream[0], so it can only start once the memory copy from
%	device to host issued to stream[0] has completed. If the code is rewritten the following
%	way (and assuming the device supports overlap of data transfer and kernel execution)
%	for (int i = 0; i < 2; ++i)
%	cudaMemcpyAsync(inputDevPtr + i * size, hostPtr + i * size,
%	size, cudaMemcpyHostToDevice, stream[i]);
%	for (int i = 0; i < 2; ++i)
%	MyKernel<<<100, 512, 0, stream[i]>>>
%	(outputDevPtr + i * size, inputDevPtr + i * size, size);
%	for (int i = 0; i < 2; ++i)
%	cudaMemcpyAsync(hostPtr + i * size, outputDevPtr + i * size,
%	size, cudaMemcpyDeviceToHost, stream[i]);
%	then the memory copy from host to device issued to stream[1] overlaps with the kernel
%	launch issued to stream[0].
%	On devices that do support concurrent data transfers, the two streams of the code
%	sample of Creation and Destruction do overlap: The memory copy from host to device
%	issued to stream[1] overlaps with the memory copy from device to host issued to
%	stream[0] and even with the kernel launch issued to stream[0] (assuming the device
%	supports overlap of data transfer and kernel execution). However, for devices of
%	compute capability 3.0 or lower, the kernel executions cannot possibly overlap because
%	the second kernel launch is issued to stream[1] after the memory copy from device
%	to host is issued to stream[0], so it is blocked until the first kernel launch issued to
%	stream[0] is complete as per Implicit Synchronization. If the code is rewritten as
%	above, the kernel executions overlap (assuming the device supports concurrent kernel
%	execution) since the second kernel launch is issued to stream[1] before the memory copy
%	from device to host is issued to stream[0]. In that case however, the memory copy from
%	device to host issued to stream[0] only overlaps with the last thread blocks of the kernel
%	launch issued to stream[1] as per Implicit Synchronization, which can represent only a
%	small portion of the total execution time of the kernel.
		 
	\subsection{nvcc compiler}
	To compile the CUDA C++ code was necessary to use a special compiler, included in CUDA Toolkit, that is \texttt{nvcc}.\\
	The CUDA compiler driver hides the details of CUDA compilation from developers. It accepts a range of conventional compiler options, for example for the project we could define macros or include library paths \footnote{NVCC documentation: https://docs.nvidia.com/cuda/cuda-compiler-driver-nvcc/index.html\#introduction}.\\ 
	All non-CUDA compilation steps are forwarded to a C++ host compiler that is supported by \texttt{nvcc}. 
	Source files compiled with \texttt{nvcc} can include a mix of host code and device code. \texttt{nvcc}'s basic
	workflow consists in separating device from host code and then:
	\begin{itemize}
		\item compiling the device code into an assembly form (PTX code) and/or binary form (cubin object), \item and modifying the host code by replacing the \texttt{<<<...>>>} syntax introduced in	Kernels \footnote{See  \hyperref[subs:ker]{subsection 2.3.1}} by the necessary CUDA C runtime function calls, to load and launch each compiled kernel from the PTX code and/or cubin object.
	\end{itemize}
	The modified host code is output either as C code that is left to be compiled using another tool, or as object code directly by letting \texttt{nvcc} invoke the host compiler during	the last compilation stage.
	Applications can then either link to the compiled host code (most common case), or ignore the modified host code (if any) and use the CUDA driver API to load and execute the PTX code or cubin object.\\
	This compiler can be used almost the same way as a classic \texttt{gcc}, for example:
	
	\texttt{nvcc -std=c++14 -g -G -o executable source.cu}\\	
	Here we present compiler versions installed in our two machines:	
	\begin{itemize}
		\item \textbf{Tesla P100}\\
		nvcc: NVIDIA (R) Cuda compiler driver, 
		%Copyright (c) 2005-2019 NVIDIA Corporation
		%Built on Wed_Apr_24_19:10:27_PDT_2019
		%Cuda compilation tools,
	 release 10.1, V10.1.168			
		 
		\item \textbf{Tesla M40}\\
	 nvcc: NVIDIA (R) Cuda compiler driver, 
	 %Copyright (c) 2005-2019 NVIDIA Corporation
	 %Built on Fri_Feb__8_19:08:17_PST_2019
	 %Cuda compilation tools, 
	 release 10.1, V10.1.105
	\end{itemize}
		 
	\subsection{\texttt{cuda-gdb} debugger}
	\texttt{CUDA-GDB} is the NVIDIA tool for debugging CUDA applications (available on Linux). It is an extension to the x86-64 port of GDB, the GNU Project debugger.\\
	Cuda-gdb main features are:
	\begin{itemize}
		\item it gives environment that allows simultaneous debugging of both GPU and CPU code within the same application;
		
		\item as programming in CUDA C is an extension to C programming, debugging with CUDA-GDB is an extension to debugging with GDB, so the existing GDB debugging features are present for debugging the host code (additional features have been provided to support  CUDA device code);	
		
		\item it allows to set breakpoints, to single-step CUDA applications, and also to inspect and modify the memory and variables of any given thread running on the hardware;
		
		\item it supports debugging all CUDA applications, whether they use the CUDA driver API, the CUDA runtime API, or both.
		
		\item it supports debugging kernels that have been compiled for specific CUDA architectures, but also supports debugging kernels compiled at runtime, referred to as just-in-time (JIT) compilation.
	\end{itemize}	
	\texttt{CUDA-GDB} was used to debug all compiled source files, both \texttt{.cu} and \texttt{.cpp} .
	Mainly it was really helpful in this project to step device code, inspect runtime errors thrown by CUDA API calls and check exactly what code was doing inside our kernels.
	
\section{Visual Studio Code}
	Visual Studio Code is an open-source code editor by Microsoft for Linux Operative Systems too. It includes support for debugging, embedded Git control and GitHub, syntax highlighting, intelligent code completion, snippets, and code refactoring \footnote{Other infos: https://en.wikipedia.org/wiki/Visual\_Studio\_Code \\
		Documentation: https://docs.microsoft.com/it-it/dotnet/core/tutorials/with-visual-studio-code \\ 
		Website: https://code.visualstudio.com/}.\\
	Since it's customizable we could add C++ and CUDA editor extensions, furthermore an SFTP extension allowed us to quickly upload/download files from remote machines. 	
	
\section{Tests, Result gathering, Plots}
Some other peripheral tools were involved in this project.
In particular we needed:
\begin{itemize}
	\item a way to serially run executions of our applications,possibly varying input dataset on interest values;
	\item put all results on text files;
	\item implement a script to compute averages, speedups and other interest metrics, from results on text files;
	\item a generator of graphs on important values from the obtained calculations.
\end{itemize}

\subsection{Bash scripts}
\label{subs:bash}
Bash (Bourne-Again SHell) is the shell, or command language interpreter, for the GNU operating system; the latter provides other shells, but Bash is the default  shell\footnote{https://www.gnu.org/software/bash/manual/}. 

\textbf{\texttt{Bash scripts}} were needed to implement tests, that run more executions of a certain CUDA application, also varying input dataset.
We programmed bash scripts (\texttt{.sh}) tests to contain several command as compile a certain CUDA application, run many times the related executable, profile it with \texttt{nvprof} and so on.

\subsection{Python scripts}
As Python \footnote{The Python version used to compile, on local host, our scripts is: Python 2.7.6.\\ See documentation: https://docs.python.org/2/} has dynamic typing, together with its interpreted nature, it's an ideal language for scripting and rapid application development.

In the case of this work was most useful to quickly implement a result filter: given text files with time measures, we computed averages and some speedups. \\
Furthermore we implemented a script to generate some plots on averages and speedups, exploiting the library \texttt{matplotlib} \footnote{https://matplotlib.org/}





    %\section{Project Logic}
\chapter{Project Logic}
\label{chap:logic}
%\pagenumbering{arabic}

In this work we started considering the features and problems recognizable as a Farm parallel pattern.
Then the study moved to consider how a GPU works, its main architectural characteristics and facing its data parallel nature.
Next we had to think how to "merge" two such different behaviors, in order to reach reasonable performances\footnote{About expected performances see \hyperref[sect:overallLogic]{Section 3.3} and \hyperref[sect:tunings]{Section 3.4} for more clarifications.}, i.e. almost competitive with a classic data parallel problem.
Finally, once the main idea behind the development was clear, we had to make some tunings.
All of these steps will be shown in detail in next sections.

\section{Stream Parallelism: Farm pattern}

	Stream parallel patterns describe problems exploiting parallelism among computations relative to different independent data items, appearing on the program input stream at different times.
	Each independent computation ends with the delivery of one single item on the program output stream.\\
	We focused on \textbf{Farm parallel pattern}, modeling embarrassingly parallel stream parallelism.\\
	The only functional parameter of a farm is the function \(f\) needed to compute the single task\cite{spm}.
	Given a stream of input tasks 
	\begin{center}
		\(x_m , . . . , x_1\)\\
	\end{center}
	the farm with function \(f\) computes the output stream as
	\begin{center}
		\(f ( x_m ), . . . , f ( x_1 )\)
	\end{center}
	Its parallel semantics ensures it will process the single task in a time close to the time needed to compute \(f\) sequentially. The time between the delivery of two different task results, instead, can be made close to the time spent to compute f sequentially divided by the number of parallel agents used to execute the farm, i.e. its parallelism degree\cite{spm,parpattbench}.\\


The correspondent task farm skeleton, with a parallelism degree parameter, may therefore be defined with
the high order function description:\\
\texttt{let rec farm f =}
	
	\texttt{function}
	
	\texttt{EmptyStream -> EmptyStream}
	
	\texttt{| Stream(x,y) -> Stream((f x),(farm f y));;}\\\\
	%
whose type is

\( farm :: (\alpha \rightarrow \beta) \rightarrow \alpha \  stream \rightarrow \beta \ stream \)\\\\
%
%
\textit{The parallel semantics, associated to the higher order functions, states that the computation of any item appearing on the input stream may be performed in parallel}, according to the number of available workers\cite{spm}.\\

In the farm case, according to the parallel semantics a number of parallel agents, computing function \(f\) onto input data items, equal to the number of items appearing onto the input stream could be used. This is not realistic, however, for two different reasons:

\begin{enumerate}
	\item items in the stream do not exist all at the same time, since a stream is not a vector.
	Items of the stream do appear at different times. Actually, when we talk of consecutive items \(x_{i}\) and \(x_{i+1}\) of the stream we refer to items appearing onto the stream at times \(t_{i}\) and \(t_{i+1}\) with \(t_{i} < t_{i+1}\). As a consequence, it makes no sense to have a distinct parallel agent for all the items of the input stream, as at any given time only a fraction of the input stream will be available.
	
	\item if we use an agent to compute item \(x_{k}\), presumably the computation will end at some
	time \(t_{k}\). If item \(x_{j}\) appears onto the input stream at a time \(t_{j} > t_{k}\) this same agent can be used to compute item \(x_{j}\) rather than picking up a new agent. \\
\end{enumerate}
This is why the parallelism degree of a task farm is a critical parameter: a small parallelism degree doesn't exploit all the parallelism available (thus limiting the speedup), while a large parallelism degree may lead to inefficiencies as part of the parallel agents will be probably idle most of time\cite{spm}.\\


\subsection{Farm performance model}
\label{subs:farmperfmodel}
	It's important to consider which kind of performance indicators are useful. When dealing with performances of parallel applications, we are in general interested in two distinct kind of measures:
	\begin{itemize}
		\item those measuring the absolute (wall clock) time spent in the execution of a given (part of) parallel application. Here we can include measures such as
		\begin{enumerate}
			\item \textbf{Latency} (\(L\)). The time spent between the moment a given activity receives input data and	the moment the same activity delivers the output data corresponding to the input.
			\item \textbf{Completion time} (\(T_c\)). The overall latency of an application computed on a given input data set, that is the time spent from application start to application completion.
		\end{enumerate}
		\item those measuring the throughput of the application, that is the rate achieved in the delivering of the results. Here In this case we can include measures such as
		\begin{enumerate}
			\item \textbf{Service time} (\(T_s\)). The time intercurring between the delivery of two successive output items (or alternatively, the time between the acceptance of two consecutive input items), and
			\item \textbf{Bandwidth} (\(B\)). The inverse of the service time.
		\end{enumerate}
	\end{itemize}
	These are the basic performance measures of interest in parallel/distributed computing.
	Applications with a very small service time (a high bandwidth) have a good throughput but not necessarily small latencies\cite{spm}.\\
	As an example, if we are interested in completing a computation within a deadline, we will be interested in the application completion time rather than in its service time.\\
	Each performance measure has an associated “semantics”:
	\begin{itemize}
		\item	\(L, T_c\) the latency and the completion time represent “wall clock time” spent in computing a given parallel activity. We are interested in minimizing latency when we want to complete a computation as soon as possible;
		\item \(T_s, B\) service time represents (average) intervals of time incurring between the delivery (or acceptance) of two consecutive results (or tasks). We are interested in minimizing service time when we want to have results output as frequently as possible, but we don't care of latency in the computation of a single result. As bandwidth is defined as the inverse of the service time, we are interested in bandwidth maximization in the very same cases\cite{spm}.
	\end{itemize}

	Given those measures of interest, we can describe an \textit{approximate performance model} for Farm parallel pattern.\\
	First we start considering the service time strictly for workers activity
	\begin{equation}
		T_s = \frac{T_w}{n_w}
	\end{equation}
	So in task farm, the service time is given by the service time of the workers (\(T_w\)) in the farm, divided by the number of workers (\(n_w\)), as hopefully \(n_w\) results will be output by the workers every \(T_w\) units of time\cite{spm}.
	
	We can add to this model the times spent by emitter and collector, but this will depend strictly on how the Farm is implemented. For example, suppose we have an emitter/collector single process, it gathers input tasks from the input stream and schedules these tasks for execution on one of the available workers. Workers, in turn, receive tasks and once workers compute tasks, they send back to the emitter/collector process the results\footnote{The emitter/collector process \textendash concurrently to its task scheduling	activity\textendash  collects results, possibly re-order them to respect the input task order in the output sequence too.}.
	This template is often referred to as \textit{master-worker pattern}.\\
	% although–in our perspective–this is actually a template suitable to implement both the task farm and map/data parallel skeletons.}.

	The three activities \textendash task scheduling, task computation and results gathering and dispatching \textendash happen to be executed in pipeline\footnote{A performance model of the pipeline service time states that its service time is the maximum of the service times of the pipeline stages, as it is known that the slowest pipeline stage, sooner or later, will behave as the	computation bottleneck.}. Therefore the service time of the master worker is approximated by the maximum of the service times of the three stages. However, first and third stages are executed on the same processing element (the one hosting the emitter/collector concurrent activity), so the model approximation will be
	\begin{equation}
		T_s(n) = max\{\ \frac{T_w}{n_w}, (T_e + T_c)\ \}
	\end{equation}
	Another example could be to have two distinct processing elements, one for the emitter and one for collector, so we could consider the performance model exactly as a three stages pipeline\cite{spm}. This will change the service time approximation in
	\begin{equation}
		T_s(n) = max\{\ \frac{T_w}{n_w}, T_e, T_c \ \}
	\end{equation}
	In our case in particular, we can assume that \(T_e\) is given by host to device memory copy and \(T_c\) is device to host transfer time.
	\vspace{0.3cm}
		
	In this work we mainly wanted to observe behaviors and performances in terms of completion time. We focused on decreasing as much as possible the time needed to compute a set of Stream parallel tasks on GPU.\\
	Furthermore, when modeling performance of parallel applications, we may also be interested in some derived performance measures. In this case we wanted to derive, from completion times, the speedup measures. \\
	\textbf{Speedup} is the ratio between the best known sequential execution time (on the used target architecture) and the parallel execution time. Speedup is a function of \(n\), the parallelism degree of the parallel execution\footnote{We'll see details on speedup, how it's calculated and how we introduced it in this thesis, in Chapter \ref{chap:experim}}.\\
	According to what we want to measure in Farm parallel pattern on GPU, we can give an approximation as completion time performance model.
	Assuming that sequential version is given by
	\begin{equation}\label{eq:Tseq}
		T_{seq}\approx T_{H2D} + T_{ker} + T_{D2H}
	\end{equation}
	where \(T_{ker}\) is the overall time spent in computations (kernel execution), \(T_{H2D}\) the time it takes for memory copy from host to device and \(T_{D2H}\) the time to transfer results back from device to host.\\
	Then we can approximate completion time as follows	
	\begin{equation}
	T_{comp} \approx  \frac{T_{seq}}{n_w} + T_{ov}
	\end{equation}
	Here \(n_w\) is the number of workers, \(T_{seq}\) is the time needed in sequential version given by equation \ref{eq:Tseq}, while \(T_{ov}\) is the overhead given by all those times where we can't achieve perfect overlapping (for both transfer/kernel and kernel/kernel cases).\\
	We recall that we don't know a priori the input/output stream length, it can be indefinitely long. But, the above formula means that, no matter how many tasks arrive from the input stream, the Farm should ideally divide the sequential time among all the workers, apart from (a hopefully negligible) overhead.\\ 
	A more detailed analysis will be given in experiments, Chapter \ref{chap:experim}.


	


\section{CPU-GPGPU: heterogeneous architecture}
	The target of this project was to exploit GP-GPUs high parallelism to lighten the CPU from computation intensive problems, in particular associated to the above explained Farm parallel pattern.\\
	So we had to think what could happen if we wanted to manage such computations on input/output streams, in a way such that:
	\begin{enumerate}
		\item Input stream arrives from host side, being directly generated by CPU or acquired from a remote source;
		
		\item Items are sent from host (main memory) to the GPU (global memory);
		
		\item GPU multiprocessors perform specific computations on all items of the stream (as soon as they're available in global memory);
		
		\item Finally, computed elements will be copied back to host side and will become the output stream. \\
	\end{enumerate}
	In this list our main concern is about data transfer, i.e. step 2 and 4 (as we mentioned in Section \ref{subs:gpgpustreampar}). 
	Indeed these phases introduce an overhead per se, but especially in Farm parallel pattern they can represent a not negligible bottleneck. \\
	We should not forget that the input we're handling, is a stream of items: even if elements are available with a high throughput, they come "one by one". As we mentioned in the previous section, we only have a part of input available in a certain point, so it wouldn't be realistic to have one worker per stream element.\\
	Furthermore, in our case, the items in the input/output streams are data parallel tasks, i.e. each Farm worker will have to process an embarrassingly parallel function on the input task.\\
	This means that the tasks are small collections, made up by simple elements, such as float numbers for example. Here "small collection" means that a single data parallel task has much smaller dimensions, than classical data parallel computations for GPUs\footnote{In Chapter \ref{chap:experim} we'll see applications and management of Farm streams and tasks.}.\\
	
	
	In our case we can suppose Streaming Multiprocessors to be the Farm workers so, if we don't give enough work (tasks) to each of them, we would have a resources under-utilization.\\
	Furthermore, we surely don't want to transfer from/to GPU one simple element (e.g. a single float) at time but, in some way we have to keep as much as possible the nature of a Stream parallel pattern. That's why our Farm model transfers and computes small data parallel tasks, so it's important to observe the behavior for different task sizes too(i.e. how much small should be data parallel tasks from the input stream).\\	
	As a consequence, we had to figure out some mechanisms to:
	\begin{itemize}
		\item Hide data transfer times as much as we can, both in  \(Host \rightarrow Device\)  and  \(Device \rightarrow Host\)  direction;
		\item Exploit almost completely our workers resources, in other words try to make Streaming Multiprocessors as busy as possible.
	\end{itemize}

	
	\subsection{Overlapping: Data Transfer hiding}
	In \hyperref[subs:stream]{Section 2.3.3}, we introduced CUDA Streams and we recall that they can perform asynchronous memory copies to or from the GPU	concurrently with kernel execution. 
	Since the machines we worked on, both have concurrency on data copy and kernel execution and both include 2 copy engines\footnote{Chapter \ref{chap:tools} explains copy engines utility and functioning.}, we exploited these capabilities combined with CUDA streams.\\
	In this way we aimed to achieve a situation in which we could overlap, as much as it was possible, the time it took for the GPU to execute a kernel (possibly overlapping kernels between them too) and the time it took to transfer data back and forth.
	
	As an example see Figure \ref{fig:threeStreams} to understand a simple case with 3 CUDA Streams.
	In that diagram we can see the expected behavior of three streams, but we can extend our expectations to more than three streams, without forgetting that we can have at most 2 data transfer at the same time (given that we've two copy engines).
	
	It's important to point out that not always we can have an ideal behavior in Streams, leading to a performance improvement lower than the amount we expected.\\
	Overlap amount depends on several factors: on the order in which the commands are issued to each stream, whether or not the device supports overlap of data transfer and kernel execution, concurrent kernel execution, and/or concurrent data transfers and finally the relative weight of data transfers time and kernel executions time\cite{cudabestpractices,cudaguide}.\\
	Given that our GPUs supported all kind of above mentioned mechanisms, among the above factors the  ones that can have some impact in our case are the order in which commands are issued to each stream, and the kernel/data transfer time ratio.\\
	\begin{figure}	
		\includegraphics[width=\linewidth]{images/3Streams.jpg}
		\caption{Ideal behavior for 3 CUDA Streams.}
		\label{fig:threeStreams}
	\end{figure}

	To improve the potential for concurrent kernel execution, synchronization of any kind should be delayed as long as possible\cite{cudabestpractices}.
	So, we were careful to avoid \textit{Implicit synchronizations}\footnote{Implicit synchronization automatically happens when certain host operations are issued in-between commands given from different streams. E.g. this happens in case of: pinned memory allocations, device memory allocations, non-asynchronous memory operations etc.\cite{cudastrandconcurr}.} and all unnecessary \textit{Explicit synchronizations} \footnote{In CUDA there are several command to force synchronization either between host and device, or between streams.}.
	
	Another important face of overlapping, is that it requires to balance Kernels work in such a way it's sufficient to hide the time spent in data transfers, as we quoted just above. 
	This said we can have two possibly unfair scenarios:
	\begin{itemize}
		\item Data transfers take a small amount of time, while kernels are doing lot of computations;
		
		\item Data transfers take a big amount of time, with respect to time spent in kernel execution.
	\end{itemize}
	
	The former case may arise when we have a computation-intensive application or an \textit{irregular kernel}.
	By "irregular" we mean that we're facing an inefficient kernel, due to any flow control instruction (\texttt{if}, \texttt{switch}, \texttt{do}, \texttt{for}, \texttt{while}), that can significantly affect the instruction throughput by causing threads of the same warp to diverge, i.e. to follow different execution paths.\\ 
	If this happens, the different execution paths must be serialized, increasing the total number of instructions executed for this warp. When all the different execution paths have completed, the threads converge back to the same execution path\cite{cudaguide}.	
	So we should avoid different execution paths within the same warp.\\
	However, sometimes having short memory copy and long kernels can be profitable. As an example we'll see, in Chapter \ref{chap:experim}, that having more long kernels makes them to overlap more likely both in between them and with data transfers.
	% To obtain best performance in cases where the control flow depends on the thread ID, the controlling condition should be written so as to minimize the number of divergent warps. This is possible because the distribution of the warps across the block is deterministic
	% as mentioned in Section 4.1 of the CUDA C Programming Guide. A trivial example is when the controlling condition depends only on(threadIdx / WSIZE) where WSIZEis the warp size. In this case, no warp diverges because the controlling condition is perfectly aligned with the warps 
	
	
	The latter case in the above list, may happen when we move an amount of data such that it takes a huge amount time at each transfer, w.r.t. the amount it takes in calculations. So in this case the dominant overhead factor could be the data transfer.	
	%So about these two scenarios we had to make some assumptions and tunings, that we will see in ********. 

	
	\subsection{Occupancy of GPU cores}
	Once we carried out the stream logic, we had to understand how to try to exploit almost every Streaming Multiprocessor at any given time.
	This means that we wanted to launch as many kernels as needed to exploit SMs at their best, sometimes this means to arrive near the full \textit{\textbf{Occupancy}} of an SM for each kernel execution, in other situations it's better to decrease this resource exploitation\footnote{Chapter \ref{chap:experim} will also discuss occupancy and the role it played in our applications and experiments.}.
	
	Clearly, when we start an execution, we'll have a portion of time, a sort of \textit{"warm up" phase}, where we'll have first data transfers and kernels launches. So we'll have a narrowed number of running kernels. 
	But as soon as we could have enough data transfers, and therefore enough kernels to execute, we hope to reach a workload peak on GPU.\\
	In practice, when we just said \textit{lot of kernels}, we meant a lot of small groups of (data parallel) items on which (data parallel) computations have to be applied. So, each of these items will be assigned to one or more (few) thread blocks.\\ Now we'll better explain what Occupancy means.\\

	To \textit{\textbf{maximize utilization}} the application should be structured in such a way that it exposes as much parallelism as possible and efficiently maps this parallelism to the various components of the system to keep them busy most of the time\cite{cudaguide}.\\
	The main ways to maximize utilization can be classified as follows:
	\begin{enumerate}
			\item \textbf{Application Level}
			At a high level, the application should maximize parallel execution between the host, the	devices, and the bus connecting the host to the devices, by using \textit{asynchronous functions} calls and streams;
			
			
			\item \textbf{Device Level}
			At a lower level, the application should maximize parallel execution between the multiprocessors of a device.
			Multiple kernels can execute concurrently on a device, so maximum utilization can also be achieved by using streams to enable enough kernels to execute concurrently;
			
			
			\item \textbf{Multiprocessor Level}
			At an even lower level, the application should maximize parallel execution between the	various functional units within a multiprocessor.
			In particular, \textit{a GPU multiprocessor relies on thread-level parallelism to maximize utilization of its functional units}\cite{cudaguide}. 
	\end{enumerate}
	
	From the above, we can deduce that occupancy is directly linked to the number of resident warps\footnote{We recall that warps are groups of 32 threads running in parallel on a set of instructions. Resident warps are those warps that, at a given time, are active on a certain thread block. In Chapter \ref{chap:tools} we gave a detailed explanation of these concepts.}.\\
	At every instruction issue time, a warp scheduler selects a warp that is ready to execute its next instruction, if any, and issues the relative instructions to the active threads of the warp.
	The number of clock cycles it takes for a warp to be ready to execute its next instruction is called the \textit{\textbf{latency}, and full utilization is achieved when all warp schedulers always have some instruction to issue for some warp at every clock cycle during that latency period, or in other words, when latency is completely "hidden"}\cite{perfoptimize,understandlatency}. 
	
	The most common reason a warp is not ready, to execute its next instruction, is that the instruction's input operands are not available yet.\\
	If all input operands are registers, latency is caused by register dependencies, i.e. some of the input operands are written by some previous instruction(s) whose execution has not completed yet.\\ 
	So, in this case, the latency is equal to the execution time of the previous instruction and the warp schedulers must schedule instructions for different warps during that time\cite{cudaguide,understandlatency}.\\

	Another reason a warp is not ready to execute its next instruction, is that it is waiting at some \textit{memory fence} (\textit{Memory Fence Functions}) or synchronization point.\\ A synchronization point can force the multiprocessor to stay idle as more and more warps wait for other warps in the same block to complete execution of instructions.\\
	So, having multiple resident blocks per multiprocessor can help reduce idling in this case, as warps from different blocks do not need to wait for each other at synchronization points.
	
	The \textit{number of blocks and warps residing on each multiprocessor for a given kernel call depends on the execution configuration of the call} (grid and block dimensions), the memory resources of the multiprocessor, and the resource requirements of the kernel\cite{cudaguide}.\\
	Register and shared memory are others important Occupancy variables, but we didn't focused much on them as on execution configuration. This is because, in our applications, those factors had a negligible impact on eventual occupancy.\\
%	The number of registers used by a kernel can have a significant impact on the number	of resident warps. For example, for devices of compute capability 6.x, if a kernel uses 64 registers and each block has 512 threads and requires very little shared memory, then two blocks (i.e., 32 warps) can reside on the multiprocessor since they require 2x512x64 registers, which exactly matches the number of registers available on the multiprocessor. But as soon as the kernel uses one more register, only one block (i.e.,16 warps) can be resident since two blocks would require 2x512x65 registers, which are more registers than are available on the multiprocessor. Therefore, the compiler attempts to minimize register usage while keeping register spilling (see Device Memory Accesses)	and the number of instructions to a minimum. Register usage can be controlled using the maxrregcount compiler option or launch bounds as described in Launch Bounds.
%	Each double variable and each long long variable uses two registers.

At this point, we had to reason about how and when to maximize Occupancy in our Farm parallel pattern. We shouldn't forget that after looking for full occupancy, experiments can give a prove that a lower one could be better.\\
First, we have to make some assumptions:
\begin{itemize}
	\item no shared memory was used;
	\item we took a really poor amount of registers, given the really simple nature of our example Kernels \footnote{We'll see what kind of kernels we used to test the farm parallel pattern, with some code listings in \hyperref[chap:impl]{Chapter 4}.}.
\end{itemize}  
Anyway, our chosen kernels represent some \textit{important application categories}, as we mentioned in previous chapters.\\
Then we mainly put our attention on kernel \textit{execution configuration} and number of kernels launched (by different CUDA streams), in order to try to maximize the number of active warps inside each Streaming Multiprocessor.

\subsection{Occupancy drawbacks}
Occupancy is a very important factor to take into account, but it's more important to be aware that \textit{occupancy isn't the only factor to take care of}.\\
In other words, not always trying to achieve maximum occupancy is the best idea, in some cases lower occupancy gives even better performances.\\
It is common to recommend running more threads per Streaming Multiprocessor and/or running more threads per thread block; the motivation is that this is the main way to hide \textit{latencies}.\\
Indeed, common beliefs are: multithreading is the only way to hide latency on GPU; shared memory is as fast as registers\cite{cudaguide}. Those facts aren't always true.

Some studies demonstrated how was possible to hide arithmetic latency or to hide memory latency using fewer threads, leading to code that runs faster.
The \textit{Latency} is the time required to perform an operation, for arithmetic operations it takes \(\approx20\) cycles; for memory we have \(\approx400+\) cycles instead.\\
This, in particular, means that  we can't start a dependent operation during these times, but they can be hidden by overlapping with other (independent) operations\cite{loweroccupancy}.

\begin{lstlisting}
	x= a + b; // takes about 20 cycles to execute
	y = a + c; // independent, can start anytime(stall)
	z = x + d; // dependent, must wait for completion
\end{lstlisting}
So \textit{latency hiding} means to do other operations when waiting for latency, this will make code run faster (not faster than the peak). For example another way to hide latency is \textit{Instruction Level Parallelism}\footnote{In general ILP in a kernel is intended as assigning more instructions (possibly equal between them) to each single thread, instead of having a lot of threads executing a lower amount of instructions each. Furthermore ILP can be used to execute independent instructions between two dependent ones\cite{perfoptimize}.}.\\
Furthermore another common belief is that occupancy is a metric of utilization, but, as we anticipate, it's only one of the contributing factors\cite{loweroccupancy}.

Another type of latency is memory-bounded, let's take an example:
\begin{lstlisting}
	__global__ void memcpy( float *dst, float *src){
		int block = blockIdx.x+ blockIdx.y* gridDim.x;
		int index = threadIdx.x+ block * blockDim.x;
		float a0 = src[index];
		dst[index] = a0;
	}
\end{lstlisting}
To hide memory latency, using even fewer threads, we can do more parallel work per thread:
\begin{lstlisting}
	__global__ void memcpy( float*dst, float*src){
		int iblock= blockIdx.x+ blockIdx.y* gridDim.x;
		int index = threadIdx.x+ 2 * iblock* blockDim.x;
		float a0 = src[index]; 
		//no latency stall
		float a1 = src[index+blockDim.x]; 
		//stall
		dst[index] = a0;
		dst[index+blockDim.x] = a1;
	}
\end{lstlisting}
Note: threads don't stall on memory access, they stall on data dependency instead.

Performances improve copying more floats per thread, instead of copying one and run more blocks and allocate shared memory to control occupancy\cite{loweroccupancy}.\\
For example some common concepts\footnote{From CUDA Programming Guide.} on CUDA state that:
\begin{itemize}
\item "In general, more warps are required if the ratio of the number of instructions with no off-chip memory operands to the number of instructions with off-chip memory operands is low";  %()–No, we’ve seen 87% of memory peak with only 4 warps per SM in a memory intensive kernel. 
\item “For all threads of a warp, accessing the shared memory is as fast as accessing a register as long as there are no bank conflicts between the threads”\cite{cudaguide}. 
\end{itemize}
For the former there are studies that show how a reduced quantity of warps gives good performances on memory intensive kernels.\\
For the latter, in reality shared memory bandwidth is lower than register bandwidth, in fact we should use registers to run closer to the peak.
But requiring more registers can result in having a low occupancy.\\
So, in many cases, this can be accomplished by computing multiple outputs per thread (see above example on multiple floats copy)\cite{loweroccupancy,understandlatency}.

%P100: 10.6 TeraFLOPS of peak single-precision

\section{Overall Logic}
\label{sect:overallLogic}
We have an input stream of items, in particular they are small data parallel tasks, we don't know how much they are and their arrival frequency.
The logic of this work can be summarized in the following steps:
\begin{enumerate}
	\item In the beginning, as items arrive, we start to spread them among different CUDA streams, according to a Round-Robin policy;
	\item On a certain stream, say \texttt{streams[k]}, we send out a task (e.g. the \textit{k-th} item that has arrived from the input stream) to the device (GPU Global memory);
	\item Immediately after the data transfer call, we launch the kernel execution, with a certain \textit{execution configuration}\footnote{We recall that this is given by grid and block sizes that we set for a certain kernel call with the\\ \texttt{<<< grid, block >>>} syntax.}. The kernel call will be placed in \texttt{streams[k]} as well;
	\item Once the kernel ends its computations, we copy back to host, on \texttt{streams[k]}, the result data as output item;
	\item We'll send each result element onto the output stream. 
\end{enumerate}

	\begin{figure}
		\includegraphics[width=\linewidth]{images/H2D.jpg}
		\caption{In this picture we show the schema of Farm on GPU, here we have only one task to/from the GPU.}
		\label{fig:H2D}
	\end{figure}
	
	This behavior is illustrated graphically in Figure \ref{fig:H2D}. Here we can see our input stream, from that at a given time we get a certain item, i.e. a task. In the diagram we simplified the concept of (data parallel) tasks, from the input stream, as it was a buffer of size \(k\); clearly this is only a graphical simplification. Then, we transfer the various items to Global memory of GPU.\\
	For reasons we showed in the previous sections, it would be unfeasible to work on single simple elements (e.g. float numbers) but, at the same time, we should maintain the pattern as close as possible to Stream parallel. That's why we're working on small data parallel tasks\footnote{We represented tasks as arrays, but they can be either small matrices or tiny images, as we'll see in \hyperref[chap:impl]{Chapter 4}.} of \(k\) items, where \(k\) is tested for several values but to remain a relatively small number of simple elements, possibly related to execution configuration on kernel, in particular to block size.\\

	Once we start to have several items available on GPU memory, they will be spread all over the SMs that will have enough available resources. Each task will be splitted in warps, so each active thread in a warp will compute one instruction per time, in parallel with other threads in the warp. The instruction that will be executed are essentially those specified inside kernel code.\\
	This means that we're executing in parallel the same operations over all the content in a task; in fact, as we mentioned before, each task is a small data parallel collection of items, upon which we're performing data parallel computations.  
	

	Since Figure \ref{fig:H2D} is a simplification, it may seems we're sending only one item per time to/from the GPU, this would correspond almost to a farm with one worker, processing one item per time. And this isn't completely what we wanted.\\
	\begin{figure}
		%	\hspace*{-2cm}  
		\vspace{-2cm}
		\includegraphics[scale=0.62,angle=-90]{images/overallLogic.jpg}
		\caption{Here we have an overall and broad graphical representation of our idea on how to fit a Farm parallel pattern on GPU architecture.}
		\label{fig:overallLogic}
	\end{figure}
	So, here's where CUDA Streams \footnote{Don't confuse input/output stream in Farm parallel pattern with CUDA Streams.\\ These are two completely different notions: the first is the parallel pattern input/output data type, the last are a CUDA feature (shown in \hyperref[subs:streams]{Section 2.3.3}).\\} are needed and we used them relying on the following ideas:
	\begin{enumerate}
		\item We have as many streams as Streaming Multiprocessors \footnote{Again CUDA Streams are a different concepts with respect to Streaming Multiprocessors. The first are a set of commands, the last are physical processing units inside the GPU.\\} and, at any given time, each of them hopefully issues a data transfer or a kernel executions;
		\item We should come to the point where each stream has issued at least one kernel launch, ideally we expect that each kernel execution is taken over by a certain multiprocessor. So, at a given time, we want to reach a work peak, where almost all SMs are busy;
		\item Obviously each kernel execution configuration should be well tuned, in order to take advantage of the maximum of resources in a multiprocessor (according to a certain kernel nature). 
	\end{enumerate}
	

	All of those parameters have been initially established taking into account of the NVIDIA GPUs' nature, then experimental proves have been performed\footnote{We'll see in next section more informations about Tunings.}. Measures and other estimation lead us to consider specific values for those variable parameters.\\
	So from the above facts is clear that we're exploiting each SM as a Farm parallel worker, furthermore, in such a way that all of those workers are as busy as possible. Note that our \(n_w\) \textbf{SMs-workers} apply a \textbf{kernel-function} \(f\) to all (data parallel) \textbf{tasks}.\\
	Bringing all pieces together we can summarize all project logic in Figure \ref{fig:overallLogic}.\\
	The schema may be further detailed as follows:
					
	\begin{itemize}
		\item We have \(N\) CUDA streams, where \(N\  (=n_w)\) is the number of Streaming Multiprocessors on the target machine;
		\item As input stream items arrive, in a Round-Robin way, we spread them all over the CUDA streams as follows:
		\begin{enumerate}
			\item As soon as we get the \(i^{th}\) task, it's asynchronously sent on \texttt{stream [i]} to the GPU, with the command \\ 
			\texttt{ \textbf{cudaMemcpyAsync}( devTask, hostTask, bytes, cudaMemcpyHostToDevice, \tab \tab \tab \tab stream [i]);}
			
			\item Immediately after we put kernel call, again on the \texttt{stream [i]}, to schedule the desired computations on that input item;
			
			\item Then, asynchronously again, we bring back results to host side, using the instruction \\
			\texttt{ cudaMemcpyAsync( devTask, hostTask, bytes, cudaMemcpyHostToDevice, \tab \tab \tab \tab stream[i]).}
			
		\end{enumerate}
	\end{itemize}

	
	Hopefully this approach should make each Streaming Multiprocessor (or at least a part) busy.
	Initially only few cores will be really busy, but as soon as CUDA streams get full, the pressure\footnote{In other words the amount of tasks assigned to each CUDA stream and so to each SM.} on the GPU should increase, so we expected that workload should be enough to almost fill all of Streaming Multiprocessors.\\
	\begin{wrapfigure}[20]{r}{0.4\textwidth}
		%\begin{center}
		%\raggedleft
		\centering
		\includegraphics[width=1\linewidth]{images/logicLegenda.jpg}
		%\end{center}
		\caption{Legenda about Figure \ref{fig:overallLogic}.}
	\end{wrapfigure}
	In particular, for us this translates in trying to have, at any given time, the maximum number possible of active threads, having to execute instructions, inside each SM.\\
	Figure \ref{fig:singleStream} gives a closer look to what we just said.\\
	%Observing Figure \ref{fig:overallLogic}, w
	
	From that scheme, looking at violet numbered labels, we can see the order in which we issue commands in a stream, and this will be the order in which they will be issued to device side too, for that stream. \\
	The behavior of overlapping between different streams, isn't predictable\cite{cudaguide}. Anyway we should take advantage of the fact that, considering two different CUDA streams, we can overlap data transfer and/or kernel execution in a \texttt{stream [i]} with the ones in a \texttt{stream [j]} (for some \(j\in[0,N-1], \: for \: N =\# SMs\)). Obviously, when the number of CUDA streams is greater than 3, we can have only 2 data transfer operations issued at the same time (by two distinct streams)\footnote{As mentioned in \hyperref[chap:tools]{Chapter 2}, for concurrent memory copy between host and device, we have 2 copy engines.}.
	
	Note that in the Figure \ref{fig:singleStream} we represented a single kernel execution as fully occupying an entire SM; in reality not always we'll have this behavior, sometimes it's even convenient to not fully occupy an entire SM with a single kernel launch\footnote{We'll see how we practically tried out Streaming Multiprocessors \textit{occupancy} in  \hyperref[chap:experim]{Chapter 5}}. \\
	
	Essentially if all of our reasoning and theories are right, we would expect that we can have an improvement, on completion time, roughly in the order of SMs number with respect to the \textit{serial approach}. \\
	This similarly means that if, for example, we have 3 CUDA Stream we would expect to take an advantage on only at most 3 SMs (at peak work flow), so this should give us an improvement, in completion time, of at most 3 times compared to classical approach.\\
	The case of serial approach, instead, processes input stream items without any type of overlapping, it is equivalent to:
	\begin{itemize}
		\item send input to device and wait on host for data transfer completion;
		\item call the kernel;
		\item host calls the copy back, from device to host, and keeps waiting until the end of data transfer;
		\item in the meanwhile kernel is running on GPU; 
		\item finally, only when all computations ended up, results are transferred back to the host, that at this point finishes to wait for output and, so, it can continue with next task.
	\end{itemize} 
	

	\begin{figure}%[ht!]
		%	\hspace*{-2cm}  
		%\vspace{-2cm}
		\includegraphics[scale=0.56]{images/singleStream.jpg}
		\caption{In this picture we can see what exactly happens in a certain CUDA Stream. Light violet numbered labels shows the order in which commands are issued by host to a certain stream.}
		\label{fig:singleStream}		
		
	\end{figure}	
\section{Tunings}
\label{sect:tunings}
	We showed a lot of peculiar behavior and architecture characteristics, as they were taken into account for different implementations, tests datasets and results analysis.\\
	It's clear that, once we decided how to organize our Farm parallel pattern for the GPU, we had to perform experiments and empirical evaluations.
	This is because:
	\begin{itemize}
		\item It was important to think about a general logic, that wasn't architecture-dependent\footnote{At least we can say that the overall view, showed in Fig. \ref{fig:overallLogic}, can be plausible with almost all NVIDIA architecture having more than two copy engines and allowing concurrent kernel execution.};
		
		\item To validate our idea we had to make a lot of experiments, time measures, examples and counterexamples too;
		
		\item Clearly experiments required to get a little deeper on NVIDIA GPUs architecture, considering the good practices, apart from the considered application;
		
		\item Finally, we had to consider some feature totally application-bounded to launch tests and obtain results of interest.
	\end{itemize}
	
	So after the logical phase, we went through a \textit{tuning phase}, that first had to face general NVIDIA GPUs behavior and structure\footnote{In \hyperref[chap:experim]{Chapter 5} we'll mainly see tunings based on the GPUs used to run tests \textendash \textbf{P100} and \textbf{M40}.}.
	Some of the important best practices we tried out in tuning phase were:

	\begin{itemize}
		\item The effect of execution configuration on performance for a given kernel call generally
		depends on the kernel code, so experimentation is recommended and in fact we followed that approach;
		\item The number of threads per block should be chosen as a multiple of the warp size (generally equal to 32 threads) to avoid wasting computing resources with under-populated warps\footnote{That's because kernels issue instructions in warps (groups of 32 threads). For example, if we have a block size of 50 threads, the GPU will still issue commands to 64 threads, so we would waste 14 of them idling.};
		\item We exploited \textbf{\textit{Occupancy Calculator}}\footnote{Those tools are included in CUDA Toolkit, they assist programmers in choosing thread block size based on kernel behavior, register and shared memory requirements.} both in spreadsheet and API functions\footnote{These are special function to call inside the code, we'll see in \hyperref[chap:impl]{Chapter 4} a code example on how and where they are used.} formats\cite{cudaguide}.
		 
	\end{itemize}

	Given those initial guidelines, it's important to highlight what are variable parameters in Figure \ref{fig:overallLogic}, on which the tuning was made:
	\begin{itemize}
		\item The number of CUDA streams;		
		\item The number of threads per block (block size);
		\item The number of blocks (grid size).
	\end{itemize}
	The first parameter changes mainly as the target machine changes\footnote{We'll see in Chapter \ref{chap:experim} that the experiments are performed on zero and three CUDA streams. Furthermore we test the case in which we have as many CUDA streams as SM in the target GPU, so this value will change between P100 and M40 too.}.\\
	The second parameter, in our case, generally depends both on input tasks dimensions and CUDA limits on thread blocks dimensions. But the threads per block choice may be also influenced by the kernel nature and a tuning according to performance measures.\\
	The third is generally roughly determined dividing the size of the input by the block size. However Farm is a particular case and sometimes it's useful to determine the grid size empirically.\\
	
	The \hyperref[chap:experim]{Chapter 5} shows all main parameters tested and their respective performances.
 
			 
\subsection{Tuning on block and grid dimensions}
	It's important to understand some main concepts, that are the basis for the logic of this project.\\
	As we mentioned above, the variation on \textit{thread block size} and \textit{grid size}, can affect heavily performances, especially in an extreme scenario as the case study of this thesis\footnote{Here we mention some useful discussions on occupancy, block and grid dimensions from Stack Overflow:\\
		https://stackoverflow.com/questions/9985912/\\
		https://stackoverflow.com/questions/5643178/\\
		https://stackoverflow.com/questions/54715373/ }.
	There's a tight correlation in between \textit{Occupancy}, \textit{kernel execution configuration} and kernel code nature.
	First note that a certain block, whatever its dimension is, will be run on a single Streaming Multiprocessor and once it's assigned it will never be moved. When resources are allocated for a thread block in an SM, it will become an \textit{active block}.\\
	In an SM we can have multiple blocks running independently, each of which grabbing its portion of resources, ie we can have multiple active blocks on a SM until they don't hit the maximum allowed\cite{perfoptimize,cudaguide}.
	So, inside a certain SM, some particular cases may happen:
	\begin{enumerate}
		\item Maximum block dimension (aka lot of threads per block), can lead to have a smaller number of blocks\footnote{Even only one, as it happens for one of the applications we tested. Anyway, from best practices guide and from profiling this should be a inefficient configuration, so we should be very careful on performances from these situations.};
		\item Small block dimension, can give a higher number of blocks running on the same SM\footnote{We recall that GPUs have also a limit on the number of active blocks per SM.} .
	\end{enumerate}
	
	In the first scenario, we can have cases of good performances in some situations. For example, if few blocks have lot of work to do, it's more likely that they will monopolize resources of the SM in which they're active, making all other waiting blocks scheduled in the same SM, idle for too long (for example in our GPUs it may happens for \textit{blocksize = 1024}). 
	
	In the second scenario we can have cases of really poor performances due to low resources exploitation, but for some kinds of kernel we may have a gain, especially in cases as the above mentioned monopolization. So smaller blocks size, means more active blocks on a SM, at any given time.
	
	In general, there is a performance \textit{sweet spot} for middle values on thread block size(for example usually identified in \textit{blocksize = 512}).\\
	%To better explain that concept, let's take a simple example. It's the first type of kernel we tested \footnote{We'll take a closer look on that case study, with implementation details, in \hyperref[chap:impl]{Chapter 4}.}, so suppose we have:
	%\begin{itemize}
	%	\item Vectors of floats as input and output;
	%	\item We have a "regular" kernel, in other words inside that we have not irregular workflows, we haven't divergent execution flowsand workload between all threads is almost the same;
	%	\item We avoid all types of synchronizations, both thread synchronization (\texttt{\_\_syncthreads()}) and host/device one (\texttt{cudaDeviceSynchronize()}).
	%\end{itemize}
	%In this situation
	So our tests had to face with the above explained behavior too, without forgetting the nature of the various implemented kernels and the relative latencies.


%\section{CPU/GPU Scheduling}
%\label{sect:cpugpuscheduling}
%In addition to the main logic of our project we introduced another branch of study.\\
%In particular, we extended the Farm parallel pattern on GPGPU introducing a sort of \textit{\textbf{CPU/GPU Scheduler}}.\\
%This consist in an implementation that, given an initial work percentage, it gradually and experimentally tunes those percentages to balance jobs.
%In particular, the scheduler adjusts the dimensions of data chunks directed to CPU or GPU on the basis of previous measured completion times of both devices.\\
%Clearly, starting from a user provided percentage, measured times are used to recompute percentages (and thus chunks dimension). So, in a finite number of algorithm steps, the two portion size will stabilize around two "good" values.\\
%This allow us to let host and device cooperate to apply same computations, but with different workloads. Clearly the main idea is to let GPU have a greater workload with respect to the one for CPU, so that  latter can be lighten from doing the entire computations; at the same time, having a good occupancy on GPU, we can gain a speedup compared to letting only one of the two processors doing all the work.\\
%
%	***************
%	qui dovresti dire che dipende dal tipo di computaz.
%	e in particolare dal rapport fra Tcomm e Tcalc ...
%************************



%******************
%LO SCHEMA GENERALE NON È SPIEGATO BENISSIMO.....
%******************
    \chapter{Implementation} 
\label{chap:impl}
%\pagenumbering{arabic}

Specs and code.
	


\section{Stream Parallel on GPU}
	\textbf{\(<<<\)\, blockSize\(>>>\)}
	%\subsection{Low Parallel un GPU}

\section{Data Parallel un GPU}
	\textbf{\(<<<\)dataSize/blockSize, blockSize\(>>>\)}
	
	
	\subsection{CUDA Occupancy APIs}
	%‣
	The occupancy calculator API,
	cudaOccupancyMaxActiveBlocksPerMultiprocessor , can provide an
	occupancy prediction based on the block size and shared memory usage of a kernel.
	This function reports occupancy in terms of the number of concurrent thread blocks
	per multiprocessor.
	%‣
	%‣
	Note that this value can be converted to other metrics. Multiplying by
	the number of warps per block yields the number of concurrent warps
	per multiprocessor; further dividing concurrent warps by max warps per
	multiprocessor gives the occupancy as a percentage.
	The occupancy-based launch configurator APIs,
	cudaOccupancyMaxPotentialBlockSize and
	cudaOccupancyMaxPotentialBlockSizeVariableSMem , heuristically calculate
	an execution configuration that achieves the maximum multiprocessor-level
	occupancy.
	
	The following code sample calculates the occupancy of MyKernel. It then reports the
	occupancy level with the ratio between concurrent warps versus maximum warps per
	multiprocessor
	
\section{CPU and GPU Mix}
Queue with P and Q chunk exec by respectively CPU and GPU.

    \chapter{Experiments}
Experiments bla bla
\section{What and How}
Some explanation
\section{Results}
Results
\section{Some Plots}
    \chapter{Conclusions} \label{chap:conclusions}
%\subsection{Overview and goals}
The main goal of this thesis was to experiment if a Farm parallel pattern could fit in GPU architecture and, if this was the case, how.\\
Even though a Streaming parallel pattern may seem so far from the concept of normal GPU use, we founded our attempt on the increasing and pervasive concept od General-Purpose computing.
Nowadays it's a common practice to use the high parallelism and huge computational power of GPUs as co-processors, even if it isn't strictly for graphical problems.\\
Also research moved, in last years, the focus on problems that generally are assigned CPUs. Clearly, in General Purpose (GP) it's easy to spot applications that are clearly embarrassingly parallel; we recall that GPUs are mostly well suited in data parallel approaches.\\
However, there are many others problems that are really far from data parallel. Again, GP-GPUs demonstrate a fair behavior (with some adjustments) in some of those cases.\\
So it makes perfectly sense to inspect for new non-data parallel applications to fit in GPU model, to exploit its good computation potential.

\subsection{Evaluation of the problem}
The starting point of this study was to consider and understand some main features and the functioning of a graphic processor, in particular taking into account of the organization about parallelism, threads, cores, internal memory and so on. We showed main GPU and NVIDIA CUDA characteristics, briefly introducing them in \hyperref[chap:into]{Chapter 1} and deepening on more specific concepts in \hyperref[chap:tools]{Chapter 2} and \hyperref[chap:logic]{Chapter 3}. In the latter we also showed how some best practices and considerations were exploited to evaluate, implement and then test our model. \\

Once we had an overall view on tools and NVIDIA GPUs architecture, we had the knowledge to the next step, ie to imagine a Farm parallel pattern in a graphic processor. Obviously some key problems have arisen:
\begin{itemize}
	\item Handle the difference on input/output, ie streams of items instead of data structures;
	\item Handle how to group and send data to device;
	\item Define the dimension of data chunks;
	\item How to hide the overhead due to data transfers between host and device;
	\item How to execute many "small" kernels at the same time, instead of a single "big" one;
	\item How to exploit the capabilities of the GPU at their best.
\end{itemize}
The first two points were accomplished by thinking to a system of \textit{accumulator buffers} that was sent to device as soon as they were filled by the input stream. This was mainly designed for the simple-computation kernel, but it applies to all those scenarios where we have an input stream made of simple items (eg floats, integers, etc.).\\
Instead for the other two kernels we simply had to test the Farm parallel pattern on small matrices or small images, that straightly arrived from the input stream, so they were ready to be directly sent to device.

The third point is again bounded to all applications having simple items as input stream. The dimension of buffers was determined by both \textit{empirical approach} and a study on \textit{best practices} for GPUs, as we showed in \hyperref[chap:logic]{Chapter 3}, most of this reasoning relied on \textit{occupancy} evaluations. \\
However, we also showed, how occupancy may not be a relevant factor; a lot of performances bottlenecks may depend on the kernel nature. We have to face some \textbf{latencies} that happens inside the Streaming Multiprocessors, in our study we mainly pointed out two types of bottleneck in kernels: \textbf{memory-bounded} code and \textbf{diverging flows}.\\
Those concepts are straightly linked to the problem of last point.

The fourth and fifth points are strongly related to a powerful programming technique in CUDA:\textit{\textbf{ Asynchronous calls}} and \textit{\textbf{CUDA Streams}}.\\
We recall that here asynchronous is from a device side point of view, with respect to the host. That is, host can continue executing his code, after invoking a memory copy (or any other call that is generally blocking). Any asynchronous call will be forwarded to GPU that will "silently" work, sending back eventual results to CPU.\\
This means, in general, to have some synchronization at some point. Often CUDA codes with asynchronous calls are implemented to introduce \textit{explicit synchronizations}, in order to have correct results and avoid memory overwriting.\\
We also met that problem, having a lot of CUDA streams trying to write back results at the same time, that sometimes led to overwriting data from another non-default stream.\\
This problem mainly showed up in device memory: in host side, from the beginning, we had foresee the need of sufficient host memory locations for all streams. In device side, the possible cases were:
\begin{itemize}
	\item Have a single chunk of space in global memory and use some explicit synchronization, that's the most used approach;
	\item Reserve several chunks, as many as created CUDA streams amount, and not use any kind of explicit synchronization.
\end{itemize}
 The first approach can be used in data parallel approach without introducing a big amount of overhead, but in a stream parallel context it can cause a performance drop. So we decided to use the second approach. The first impression could be to risk for a saturation in global memory, but in our case this didn't happen, since we were using relatively small chunks of data (even if they were as many as number of CUDA streams).\\
 This doesn't mean that it cannot exist any stream parallel application where synchronization may result in an advantage\footnote{Maybe to hide some other computations that are happening in the same time. Again it's a matter of experimenting according to the type of problem we're facing.}.\\
 Furthermore trying a \textit{hybrid approach} could be a starting point for future works, where hybrid means having less allocated global memory locations (than CUDA Streams number) and introduce only few explicit synchronizations.\\
 
 
 \subsection{Implementation and tests}
 Those ideas and designs were implemented as described in \hyperref[chap:impl]{Chapter 4}. We decided to implement different kind of kernels to experiment the behavior of Farm parallel pattern in different conditions.\\
 We recall that we decided to implement three types of kernels: Simple-computational, Matrix Multiplication and Image processing.\\
 The first would have been the one from which we expected a "good behavior" in terms of performances, while we expected worse completion times and speedups from the second kernel type and even from the third one.
 
 The next step has been to build tests, gather results and make the following considerations.\\
 Tests have been set up in such a way to observe the performances of our model in different situations, such as varying the chunks size and varying the pressure on CUDA Streams, ie the number of tasks globally sent on a certain non-default stream.\\
 What we wanted to mainly measure was:
 \begin{itemize}
 	\item the global time spent to "consume" an input stream by transferring data one chunk per time, do all computations of the kernel and send back results. This is what we considered the \textit{serial version} for our applications, in other words the approach without CUDA Streams;
 	
 	\item the global time spent to "consume" an input stream by overlapping more chunks transfers and computations of the kernel. This is what we considered the \textit{parallel version}, in other words the approach using CUDA Streams (three or equal to the number of SMs);
 	
 	\item the completion time of the relative data parallel version, ie assuming that all our input stream is grouped in a single data structure.
 	
 \end{itemize}
 
 \subsection{Results and considerations}
 The results we obtained are just as we expected:
 \begin{enumerate}
 	\item Simple-computational kernel showed a great ability on overlapping, clearly with some appropriate adjustments. We get the expected speedups and the version with the maximum number of CUDA Streams\footnote{That is the version with an amount of non-default streams equal to the number of Streaming Multiprocessors.} performs almost as the data parallel version;
 	
 	
 	
 	\item Matrix multiplication kernel showed a low ability on overlapping, especially as matrices size increased. We get poor speedups and the version with the maximum number of CUDA Streams performs quite bad with respect to the data parallel version;
 	
 	
 	
 	\item Image processing kernel showed an almost  inexistent ability on overlapping. We get no speedups and the version with the maximum number of CUDA Streams performs really far from the data parallel version.
 \end{enumerate}
 From those results we understand that we have the best gain when we have long computations on each single chunks. Even if host/device data transfers introduce a not-negligible overhead, in Farm pattern, we're carrying small groups of items.\\ Furthermore, these small groups aren't available all at the same time, they arrive one at time, as they're generated from an input stream.
 This leads potentially to a low data transfers overlap, just for a timing matter.\\ 
 That's why we should mostly rely on overlapping kernels, as they should lasts longer than memory copy\footnote{This isn't a rule, it just often happens, as in our applications. There may still be cases in which this statement is false.} and so we've more chances to achieve an overlap.\\
 About this consideration we recall \hyperref[fig:cosprofiling]{5.3}.
 
 In fact, most of the problems in performances appeared in memory-bound kernels, because we have a lot of memory operations, merged with a really small amount of work per thread to do. This leads to inefficient kernels, that, in any case, last too short to afford a good overlap with other kernels or transfers.\\
 Furthermore this behavior can even degrade as the portions of data sent to the GPU grows in size, as in our Matrix Multiplication case, because we had a high number of thread blocks occupying all hardware resources, for a single matrix multiplication. This led to longer kernels, but completely monopolizing Multiprocessors. \\
 
 


\subsection{Final remarks and further works}
The results and considerations just discussed in previous section, expose the following necessities for Farm parallel pattern: 
\begin{itemize}
	\item It better performs in high-computational intensity scenarios;
	\item We get the best advantages from parallelizing executions, more than memory copies;
	\item It relies on overlapping between CUDA streams, meaning it needs quite long kernels executions to hide the host latency deriving from the acquisition of items from an input stream;
	\item Kernel launches should be configured such that they don't monopolize many multiprocessors (the best would be at most one SM occupied by a single kernel call). 
\end{itemize}
The above requirements may translates in a quite challenging effort, especially in evaluating problems, experimenting and profiling performances, more than in implementation difficulty.\\
This especially holds in all those cases were we have memory-bound problems. But in this case we some chances of future workaround:
\begin{itemize}
	\item using efficient memory access patterns for GPU memory (especially needed for global);
	\item assigning more work to each thread, for example giving more instructions to execute per kernel (Instruction Level Parallelism)\cite{cudabestpractices,loweroccupancy};
	\item exploiting shared memory, it has smaller dimensions but it's much more faster than global memory.
\end{itemize} 
These stratagems may expand to a lot of further applications using Farm parallel pattern on GPUs in the future.\\
For example numerous studies have been made in matrix multiplication to optimize device global memory latencies with shared memory.
%[METTERE UNA FONTE CHE PARLA DI SHARED]. 
Other studies showed that we can give smaller kernel configurations, in order to make each threads perform several matrix multiplications\cite{loweroccupancy}, instead of computing only one element of the result matrix for each threads (as it happens in classical approach).\\
So by merging in future, those optimizations with Farm parallel pattern may give some interesting results.\\


 Given that this thesis based all hypothesis on equal chunks of work, ie on balanced workloads for each kernel, an interesting further study could be done in those scenarios treating unbalanced chunks of works, leading to different workloads among kernel launches.\\
 Suppose, for example, a scenario where the input stream sends items at fluctuating speeds, so the chunk size may be established according to a time interval instead of a predefined buffer size. This means send portions of items of unknown size to the GPU.\\
 
 Another assumption on which we based all this study was in having the ready data chunks scheduled to CUDA Streams in a Round Robin fashion. But, according to the treated problem, other scheduling techniques can be adopted and may result in far more efficient spreading of work loads between streams and so giving a better exploitation in Multiprocessors resources.
 
    
    \lstlistoflistings
    \listoftables
    
%	*********************
%	devi mettere tutti i rif a modo: autore(i), titolo, luogo di pubblicazione (e.g. casa editrice per
%	i libri, conferenza ed editore per le conf, sito web per i siti), anno per tutti i lavori. Così non
%	vanno bene (vedi sotto)
%	****************
	\begin{thebibliography}{9}
		%\bibitem{latexcompanion} 
		\bibitem{pattersonhennessy}
		D.A. Patterson, J.L. Hennessy, 
		\textit{Computer Organization and Design: The Hardware and Software Interface}, RISC-V Edition, Morgan Kaufmann, 2014
	
	
		\bibitem{fromCUtoOCL}
		Peng Du, Rick Weber, Piotr Luszczek, Stanimire Tomov, Gregory Peterson, Jack Dongarr, 
		\textit{From CUDA to OpenCL: Towards a Performance-portable Solution for Multi-platform GPU Programming}, University of Tennessee Knoxville,University of Manchester,\\ http://www.netlib.org/lapack/lawnspdf/lawn228.pdf , 2010
		
		
		\bibitem{backtrack}
		John Jenkins, Isha Arkatkar, John D. Owens, Alok Choudhary, Nagiza F. Samatova, 
		\textit{Lessons Learned from Exploring the Backtracking Paradigm on the GPU}, 2011
		
		\bibitem{pipemicrosoft}
		Microsoft, \textit{Graphics Pipeline}, \href{https://docs.microsoft.com/en-us/windows/win32/direct3d11/overviews-direct3d-11-graphics-pipeline?redirectedfrom=MSDN}{guide available here}, 2018
		
		\bibitem{rendering}
		Hujun Bao, Wei Hua, \textit{Real-Time Graphics Rendering Engine}, 2011
		
		\bibitem{cudapipe}
		NVIDIA, \textit{Pipe Utilization}, \href{https://docs.nvidia.com/gameworks/content/developertools/desktop/analysis/report/cudaexperiments/kernellevel/pipeutilization.htm}{documentation available here}, 2015
		
		
		\bibitem{spm}
		Marco Danelutto,
		\textit{Distributed Systems: Paradigms and Models}, 2014
		
		
		
		\bibitem{cpugpumix}
		T. Serban , M. Danelutto , M. Coppola, \textit{Data parallel patterns on
			CPU/GPU mix}, Dept. Computer Science Univ. of Pisa, ISTI C.N.R. Pisa, 2012
	
		\bibitem{streamparpatt}
		J. Daniel Garcia, D. del Rio, M.F. Dolz, J. Garcia-Blas, L.M. Sanchez, M. Danelutto, M. Torquati, \textit{Stream parallelism patterns}, Computer Science Dept. University of Pisa, Computer Science and Engineering Dept. University Carlos III of Madrid,\\
		http://www.open-std.org/Jtc1/sc22/wg21/docs/papers/2016/p0374r0.pdf , 2016
		
		%\bibitem{com}
		 %David A. Patterson, John L. Hennessy, \textit{Computer Organization and Design: The Hardware/Software Interface}, Fourth Edition, Elsevier, 2012
		 
		 
		\bibitem{cudaguide}
		NVIDIA, \textit{CUDA C Programming Guide}, CUDA Toolkit Documentation, from https://docs.nvidia.com/cuda/cuda-c-programming-guide/index.html, 2019
		
		
		\bibitem{profilersguide}
		NVIDIA, \textit{NVIDIA Profilers Guide}, CUDA Toolkit Documentation, from
		https://docs.nvidia.com/cuda/profiler-users-guide/index.html , 2019 
		
		
		\bibitem{nvprofarticle}
		Mark Harris, 
		\textit{CUDA Pro Tip: nvprof is Your Handy Universal GPU Profiler}, NVIDIA Developer Blog, article from https://devblogs.nvidia.com/cuda-pro-tip-nvprof-your-handy-universal-gpu-profiler/, 2013
		
		\bibitem{loweroccupancy}
		Vasily Volkov,
		\textit{Better Performance at Lower Occupancy}, slide show from: https://www.nvidia.com/content/GTC-2010/pdfs/2238\_GTC2010.pdf , 2010
		
		
		\bibitem{understandlatency}
		Vasily Volkov, \textit{Understanding Latency Hiding on GPUs}, \href{https://www2.eecs.berkeley.edu/Pubs/TechRpts/2016/EECS-2016-143.pdf}{Technical report available here}, 2016
		
		
		\bibitem{devblogevents}
		Mark Harris, \textit{How to Implement Performance Metrics in CUDA C/C++}, NVIDIA Developer Blogs,  \\
		\href{https://devblogs.nvidia.com/how-implement-performance-metrics-cuda-cc/}{post from NVIDIA Developer Blog available here}, 2012 
		
		\bibitem{libevents}
		NVIDIA, \textit{NVIDIA Library Documentation- Event Management}, YEAR
		
		\bibitem{structparprog}
		M. McCool, A.D. Robinson, J. Reinders, \textit{Structured Parallel Programming: Patterns for Efficient Computation}, 2012
		
		\bibitem{cudabestpractices}
		NVIDIA, \textit{CUDA C Best Practices Guide}, \href{https://docs.nvidia.com/cuda/cuda-c-best-practices-guide/index.html}{CUDA Toolkit Documentation, available here} , 2019
		
		\bibitem{parpattbench}
		M. Danelutto, T. De Matteis, D. De Sensi, G. Mencagli, M. Torquati, \textit{P3ARSEC: Towards Parallel Patterns Benchmarking}, \href{http://pages.di.unipi.it/desensi/assets/pdf/2017\_SAC.pdf}{pdf paper here}, 2017
		
		\bibitem{custreamsblog}
		Mark Harris, \textit{How to Overlap Data Transfers in CUDA C/C++}, \href{https://devblogs.nvidia.com/how-overlap-data-transfers-cuda-cc/}{post from NVIDIA Developer Blog here}, 2012
		
		\bibitem{nvccdoc}
		NVIDIA, \textit{CUDA Compiler Driver NVCC}, CUDA Toolkit Documentation:\\ 
		https://docs.nvidia.com/cuda/cuda-compiler-driver-nvcc/index.html , 2019
		
		\bibitem{cudagdbdoc}
		NVIDIA, \textit{CUDA-GDB: CUDA debugger}, CUDA Toolkit Documentation:\\ 
		https://docs.nvidia.com/cuda/cuda-gdb/index.html , 2019
		
		\bibitem{cudahandbook}
		Nicholas Wilt, \textit{The CUDA Handbook: A Comprehensive Guide to GPU Programming}, 2013
		
		
		\bibitem{p100whitepaper}
		NVIDIA, \textit{NVIDIA Tesla P100}, \href{https://images.nvidia.com/content/pdf/tesla/whitepaper/pascal-architecture-whitepaper.pdf}{The technical whitepaper is available here}, 2016
		
		
		\bibitem{cudastrandconcurr}
		Steve Rennich, NVIDIA, \textit{CUDA C/C++ Streams and Concurrency}, 		\href{https://developer.download.nvidia.com/CUDA/training/\\
			StreamsAndConcurrencyWebinar.pdf}{slideshow here}, 2011
		
		
		\bibitem{perfoptimize}
		Paulius Micikevicius, NVIDIA, \textit{Performance Optimization: Programming Guidelines and GPU Architecture Reasons Behind Them}, \href{http://on-demand.gputechconf.com/gtc/2013/presentations/S3466-Programming-Guidelines-GPU-Architecture.pdf}{slide show here}, 2013
		
		\bibitem{rooflinepaper}
		S. Williams, A. Waterman, D. PAtterson, \textit{Roofline:  An insightful Visual Performance model for multicore Architectures}, \href{http://citeseerx.ist.psu.edu/viewdoc/download?doi=10.1.1.156.756\&rep=rep1\&type=pdf}{pdf paper here}, 2009
		
		\bibitem{rooflineslides}
		S. Williams, D. PAtterson,\textit{The Roofline Model: A pedagogical tool for program analysis andoptimization}, \href{https://crd.lbl.gov/assets/pubs\_presos/parlab08-roofline-talk.pdf}{slide show here}, 2008
		
		\bibitem{applyroofline}
		G. Ofenbeck, R. Steinmann, V. Caparros, D. G. Spampinato, M. Püschel, \textit{Applying the roofline model},  \href{http://spiral.ece.cmu.edu:8080/pub-spiral/pubfile/ispass-2013\_177.pdf}{paper in pdf format here}  , 2014
		
		\bibitem{optimizingcuda}
		NVIDIA corporation, \textit{Optimizing CUDA}, \href{http://developer.download.nvidia.com/CUDA/training/NVIDIA_GPU_Computing_Webinars_CUDA_Optimization_April-2009.pdf}{slides how from CUDA developer download, available here}, 2009
		
		%\bibitem{achievedflops}
		%NVIDIA corporation, \textit{Achieved FLOPs}, \href{https://docs.nvidia.com/gameworks/content/developertools/desktop/analysis/report/cudaexperiments/kernellevel/achievedflops.htm}{article how from CUDA developer download, available here}, 2015
		
	%	\bibitem{flopsbenchmarking}
	%	NVIDIA corporation, \textit{Achieved FLOPs}, \href{https://latkin.org/blog/2014/11/09/a-simple-benchmark-of-various-math-operations/}{article how from CUDA developer download, available here}, 2015
		
		
		
		
	%	https://docs.nvidia.com/gameworks/content/developertools/desktop/analysis/report/cudaexperiments/kernellevel/achievedflops.htm
	\end{thebibliography}
   % \begin{appendices}
    %    \input{chaps/appendix/appendix}
    %\end{appendices}

   % \printbibliography[heading=bibintoc,title={References}]
\end{document}